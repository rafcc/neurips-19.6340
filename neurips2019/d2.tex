%==========================================================================
%Template File for Evolutionary Computation Symposium
%==========================================================================
\documentclass{article}
\usepackage{bm}

\usepackage{amsmath,amssymb}

\begin{document}
\section{$D=2$ formula}
\if0
上で計算したように
\begin{align}
&\bm P_\text{OLS}^{(1)}
(\bm P_\text{OLS}^{(1)})^\top
=
\sigma^2 [ (\bm Z^{(1)})^\top \bm Z^{(1)} ]^{-1}
\\
&\bm P_\text{OLS}^{(2)}
(\bm P_\text{OLS}^{(1)})^\top
=
- \sigma^2 [(\bm Z^{(2)})^\top \bm Z^{(2)}]^{-1}
[ (\bm Z^{(2)})^\top \bm Z^{(2)[1]} ]
[ (\bm Z^{(1)})^\top \bm Z^{(1)} ]^{-1}
\end{align}
同様にして
\begin{align}
&\bm P_\text{OLS}^{(1)}
(\bm P_\text{OLS}^{(2)})^\top
=
- \sigma^2 [(\bm Z^{(1)})^\top \bm Z^{(1)}]^{-1}
[ (\bm Z^{(1)})^\top \bm Z^{[1](2)} ]
[ (\bm Z^{(2)})^\top \bm Z^{(2)} ]^{-1}
\end{align}
%%%
\begin{align}
\bm P_\text{OLS}^{(2)}
(\bm P_\text{OLS}^{(2)})^\top
&=
(-)^{1}
\bm P_\text{OLS}^{(2)}
\Big( (\bm Y^{(2)})^\top + (\bm P_\text{OLS}^{(1)})^\top \bm Z^{[1](2)}  \Big)
(\bm Z^{(2)})
[(\bm Z^{(2)})^\top \bm Z^{(2)}]^{-1}
\notag \\
&= %%%%
- \bm P_\text{OLS}^{(2)}(\bm Y^{(2)})^\top (\bm Z^{(2)}) [(\bm Z^{(2)})^\top \bm Z^{(2)}]^{-1}
- \bm P_\text{OLS}^{(2)} (\bm P_\text{OLS}^{(1)})^\top [\bm Z^{[1](2)} \bm Z^{(2)}] [(\bm Z^{(2)})^\top \bm Z^{(2)}]^{-1}
\notag \\
&= %%%%
\sigma^2 [(\bm Z^{(2)})^\top \bm Z^{(2)}]^{-1}
\notag \\
&\quad %%%%
+
\sigma^2 
[(\bm Z^{(2)})^\top \bm Z^{(2)}]^{-1}
[ (\bm Z^{(2)})^\top \bm Z^{(2)[1]} ]
[ (\bm Z^{(1)})^\top \bm Z^{(1)} ]^{-1}
[\bm Z^{[1](2)} \bm Z^{(2)}] [(\bm Z^{(2)})^\top \bm Z^{(2)}]^{-1}
\end{align}
となるので、それぞれの漸近値は
\fi
%%%%%  ここまでコメントアウト
Each control point matrix has the following asymptotic form:
\begin{align}
&
\bm P_\text{OLS}^{(1)}
(\bm P_\text{OLS}^{(1)})^\top
\approx
+
\sigma^2 L \Big(
\frac{1}{N^{(1)}}
(\bm \Lambda ^{(1)[1]})^{-1}
\Big)
\\
&
\bm P_\text{OLS}^{(2)}
(\bm P_\text{OLS}^{(1)})^\top
\approx
- \sigma^2 L \Big(
\frac{1}{N^{(1)}}
(\bm \Lambda ^{(2)[2]})^{-1}
(\bm \Lambda ^{(2)[1]})
(\bm \Lambda ^{(1)[1]})^{-1}
\Big)
\\
&
\bm P_\text{OLS}^{(1)}
(\bm P_\text{OLS}^{(2)})^\top
\approx
- \sigma^2 L \Big(
\frac{1}{N^{(1)}}
(\bm \Lambda ^{(1)[1]})^{-1}
(\bm \Lambda ^{[1](2)})
(\bm \Lambda ^{(2)[2]})^{-1}
\Big)
\\
&
\bm P_\text{OLS}^{(2)}
(\bm P_\text{OLS}^{(2)})^\top
\approx
+
 \sigma^2 L \Big(
 \frac{1}{N^{(2)}} (\bm \Lambda ^{(2)[2]})^{-1}
 +
 \frac{1}{N^{(1)}}
(\bm \Lambda ^{(2)[2]})^{-1}
(\bm \Lambda ^{(2)[1]})
(\bm \Lambda ^{(1)[1]})^{-1}
(\bm \Lambda ^{[1](2)})
(\bm \Lambda ^{(2)[2]})^{-1}
 \Big)
\end{align}
ただし
\begin{align}
&({\bm \Lambda}^{(m)[k]})_{ \bm d ^{(m)} \bm d ^{(k)} }
= %%%%
\frac{(m-1)!}{\ _{M} C_m}
\begin{pmatrix}
D \\
{\bm d}^{(m)}
\end{pmatrix}
%
\begin{pmatrix}
D \\
{\bm d}^{(k)}
\end{pmatrix}
%
\sum_{  |I| = m  }
\frac{
\delta_{{\bm I}, (\bm d ^{(m)} +  \bm d ^{(k)} )_{01} }
\prod_{I_i = 1}( \bm d ^{(m)} +  \bm d ^{(k)}  )_i !}{
[\sum_{I_i = 1} ( \bm d ^{(m)} +  \bm d ^{(k)}  )_i 
+ m - 1]!
}
\\
&
{\bm \Lambda}^{[k](m)}
=
({\bm \Lambda}^{(m)[k]})^\top
\end{align}
\normalsize
where $\bm d^{(m)}$ means indices of $(m-1)$-dimensional subsimplices. In this $D=2$ case, they take values like
\begin{align}
&
\bm d^{(1)} \in \Big\{
\underbrace{ [1, 0, 0 \dots, 0] }_\text{M個},
[0,1, 0, \dots, 0], \dots, [0, 0, 0, \dots, 1]
\Big\}
\label{7}
\\
&
\bm d^{(2)} \in \Big\{
\underbrace{ [1, 1, 0 \dots, 0, 0] }_\text{M個},
[0,1, 1, \dots, 0, 0], \dots, [0, 0, 0,\dots, 1, 1]
\Big\}
\label{8}
\end{align}
%%
$\delta_{{\bm I}, (\bm d ^{(m)} +  \bm d ^{(k)} )_{01} }$ is a kind of delta function defined by
\begin{align}
\delta_{{\bm I}, (\bm d ^{(m)} +  \bm d ^{(k)} )_{01} }
&=
\left\{ \begin{array}{ll}
1 & ( {\bm I} = (\bm d ^{(m)} +  \bm d ^{(k)} )_{01}  ) \\
0 & \text{otherwise}\\
\end{array} \right.
.
\end{align}
As one can see, $\bm d^{(1)}$ is a vector of only one component 1 and all zeros after,  $\bm d^{(2)}$ is a vector of only two components 1 and all zeros after.
$(\bm d)_{01}$ is defined as
\begin{align}
[(\bm d)_{01}]_i
=
\left\{ \begin{array}{ll}
1 & \text{if $\bm d_i \neq 0$} \\
0 & \text{if $\bm d_i = 0$}\\
\end{array} \right.
\end{align}
By using these ingredients, the asymptotic risk for $D=2$ inductive-skeleton fitting is represented by
\begin{align}
R_{N^{(1)}, N^{(2)} }
\approx
\sum_{\bm d_A \bm d_B} \Big(
\Sigma \circ
(\bm P_\text{SKE} \bm P_\text{SKE}^\top )
\Big)_{\bm d_A \bm d_B}
\end{align}
where $\circ$ means Hadamard product of two matrices defined as
\begin{align}
(A \circ B)_{ij}
=
A_{ij} B_{ij}
\end{align}
and 
\begin{align}
&\Sigma_{\bm d_A \bm d_B}
=
\frac{(2D)!(M-1)!}{(2D+M-1)!}
 \binom{D}{\bm d_A}
 \binom{D}{\bm d_B}
\binom{2D}{\bm d_A + \bm d_B}^{-1}
, \label{9}
\\
&
(\bm P_\text{SKE} \bm P_\text{SKE}^\top )
=
\begin{pmatrix}
\bm P_\text{OLS}^{(1)}
(\bm P_\text{OLS}^{(1)})^\top
&
\bm P_\text{OLS}^{(1)}
(\bm P_\text{OLS}^{(2)})^\top
\\
\bm P_\text{OLS}^{(2)}
(\bm P_\text{OLS}^{(1)})^\top
&
\bm P_\text{OLS}^{(2)}
(\bm P_\text{OLS}^{(2)})^\top
\end{pmatrix}
.
\label{10}
\end{align}
Please note that the order of indices $\bm d$ in \eqref{9} and \eqref{10} should be in the same order.
In other words, the matrix \eqref{9} should take the following form:
\begin{align}
\begin{pmatrix}
\eqref{7} \eqref{7}\text{components} & \eqref{7} \eqref{8}\text{components} \\
\eqref{8} \eqref{7}\text{components} & \eqref{8} \eqref{8}\text{components}
\end{pmatrix}
.
\end{align}





%------------
%以下,参考文献に関する記述
%------------

%------------------------------------------------------------------------------
\end{document}