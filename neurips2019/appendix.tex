\documentclass{article}
% if you need to pass options to natbib, use, e.g.:
%     \PassOptionsToPackage{numbers, compress}{natbib}
% before loading neurips_2019

% ready for submission
 \usepackage[preprint]{neurips_2019}

% to compile a preprint version, e.g., for submission to arXiv, add add the
% [preprint] option:
%     \usepackage[preprint]{neurips_2019}

% to compile a camera-ready version, add the [final] option, e.g.:
%     \usepackage[final]{neurips_2019}

% to avoid loading the natbib package, add option nonatbib:
%     \usepackage[nonatbib]{neurips_2019}

%%%%%%%%%%%%
%\usepackage[color]{showkeys}%ラベルを表示
\usepackage[utf8]{inputenc} % allow utf-8 input
\usepackage[T1]{fontenc}    % use 8-bit T1 fonts
\usepackage{url}            % simple URL typesetting
\usepackage{booktabs}       % professional-quality tables
\usepackage{amsfonts}       % blackboard math symbols
\usepackage{nicefrac}       % compact symbols for 1\2, etc.
\usepackage{microtype}      % microtypography
\usepackage{graphicx}
\usepackage{ascmac}
\usepackage{amsmath,amssymb,amsfonts,array}
\usepackage{theorem}
\usepackage{bm}
\usepackage{braket}
\usepackage{extarrows}
\usepackage{booktabs}
\usepackage{algorithm,algorithmic}
\usepackage{tikz}
\usepackage{color}
\usepackage{xcolor}
\usepackage{subcaption}
\usepackage{multirow}
\usepackage{xr-hyper}
\usepackage{hyperref}
\usepackage[capitalise,noabbrev]{cleveref}

%%%% for xy-hyper
\makeatletter
\newcommand*{\addFileDependency}[1]{% argument=file name and extension
  \typeout{(#1)}
  \@addtofilelist{#1}
  \IfFileExists{#1}{}{\typeout{No file #1.}}
}
\makeatother
 
\newcommand*{\myexternaldocument}[2][]{%
    \externaldocument[#1]{#2}%
    \addFileDependency{#2.tex}%
    \addFileDependency{#2.aux}%
}

%%%% for showkey of cleveref
% \makeatletter
%   \SK@def\Cref#1{\SK@\SK@@ref{#1}\SK@Cref{#1}}%
%   \SK@def\cref#1{\SK@\SK@@ref{#1}\SK@cref{#1}}%
% \makeatother

% \definecolor{refkey}{rgb}{0.9451,0.2706,0.4941}
% \definecolor{labelkey}{rgb}{0.9451,0.2706,0.4941}

%%%% for tikz
\usetikzlibrary{positioning}

%%%% for cleveref
\crefname{equation}{}{}

\theoremstyle{plain}
%\theoremstyle{definition}
\newtheorem{theorem}{Theorem}
\newtheorem{proposition}{Proposition}
\newtheorem{lemma}{Lemma}
\newtheorem{definition}{Definition}
\newtheorem{remark}{Remark}
\newtheorem{example}{Example}
\newtheorem{corollary}[theorem]{Corollary}
\newtheorem{proof}{Proof}
\renewcommand{\theproof}{}

\newtheorem{thm}{Theorem}[section]
\newtheorem{claim}{Claim}
\newtheorem{lem}{Lemma}[section]
\newtheorem{prop}{Proposition}[section]
\newtheorem{cor}{Corollary}[section]
\newtheorem{defn}{Definition}[section]
\newtheorem{exmp}{Example}[section]
%\renewcommand*{\proofname}{Proof}

%%%% commands
\newcommand{\norm}[1]{\left\| #1 \right\|}
\newcommand{\abs}[1]{\left| #1 \right|}
\newcommand{\card}[1]{\left| #1 \right|}
\newcommand{\paren}[1]{\left( #1 \right)}
\newcommand{\sqbra}[1]{\left[#1 \right]}
\newcommand{\inprod}[2]{\left\langle #1,\ #2 \right\rangle}
\newcommand{\R}{\mathbb{R}}
\newcommand{\N}{\mathbb{N}}
\newcommand{\E}{\mathbb{E}}
\newcommand{\diff}{\mathrm{d}}
\newcommand{\ZZ}{\Sigma}

\DeclareMathOperator{\rank}{rank}

\newcommand{\argmax}{\mathop{\rm arg~max}\limits}
\newcommand{\argmin}{\mathop{\rm arg~min}\limits}


\frenchspacing  %Required

% The \author macro works with any number of authors. There are two commands
% used to separate the names and addresses of multiple authors: \And and \AND.
%
% Using \And between authors leaves it to LaTeX to determine where to break the
% lines. Using \AND forces a line break at that point. So, if LaTeX puts 3 of 4
% authors names on the first line, and the last on the second line, try using
% \AND instead of \And before the third author name.

\author{%
  Akinori ~Tanaka%\thanks{Use footnote for providing further information
    %about author (webpage, alternative address)---\emph{not} for acknowledging
    %funding agencies.}
    \\
  RIKEN AIP, Keio University\\
  \texttt{akinori.tanaka@riken.jp} \\
  % examples of more authors
   \And
   Akiyoshi Sannai \\
   RIKEN AIP, Keio University\\
   \texttt{akiyoshi.sannai@riken.jp} \\
   \AND
   Ken Kobayashi \\
   Fujitsu Laboratories LTD., RIKEN AIP, Tokyo Tech\\
   \texttt{ken-kobayashi@fujitsu.com} \\
   \And
   Naoki Hamada \\
   Fujitsu Laboratories LTD., RIKEN AIP\\
   \texttt{hamada-naoki@fujitsu.com} \\
   %\And
   %Coauthor \\
   %Affiliation \\
   %Address \\
   %\texttt{email} \\
}

\myexternaldocument[main:]{main}
\title{Appendix of Paper \#6340}
\begin{document}
\maketitle

\begin{abstract}
In this appendix, we provide the proof of the main theorem, the detailed experimental settings for reproducibility, and additional experimental results.
\end{abstract}

\appendix
\section{Proof of \texorpdfstring{\cref{main:thm:risk}}{theorem 2}}\label{sec:proof-of-main-theorem}
As we commented in the main body of the paper, we can regard $\bm b (\bm t) - \hat{\bm b} (\bm t | S_N)$ as a new B\'ezier simplex, and we call its control points as $\bm p'_{\bm d}$.
By using it, the risk $R_N = \E_{S_N} \sqbra{\E_{\bm t} \sqbra{\norm{\bm b (\bm t) - \hat{\bm b} (\bm t | S_N)}^2}}$ can be rewritten as
\begin{align}
R_N
&=
\E_{S_N}
\sqbra{
\E_{\bm t}
\sqbra{
\norm{
\sum_{\bm d \in \N_D^M}
\binom{D}{\bm d}
\bm t^{\bm d} \bm p_{\bm d}'
}^2
}
}
\notag \\
&=
\E_{S_N}
\sqbra{
\E_{\bm t}
\sqbra{
\sum_{\bm d_A, \bm d_B \in \N_D^M}
\binom{D}{\bm d_A}
\binom{D}{\bm d_B}
\bm t^{\bm d_A + \bm d_A } 
\paren{
\bm p_{\bm d_A}'
\cdot
\bm p_{\bm d_B}'
}
}
}
\notag \\
&= %%%%
\sum_{\bm d_A, \bm d_B \in \N_D^M}
\binom{D}{\bm d_A}
\binom{D}{\bm d_B}
\E_{\bm t}
\sqbra{
\bm t^{\bm d_A + \bm d_A } 
}
\E_{S_N}
\sqbra{
\paren{
\bm p_{\bm d_A}'
\cdot
\bm p_{\bm d_B}'
}
}
.
\label{eqn:RN_in_supp}
\end{align}
The inner product in the expectation over $S_N$ is equivalent to $\paren{\bm P \bm P^\top}_{\bm d_A \bm d_B}$ where $\bm P$ is the control point matrix defined in main body.
So we need to massage the expectation over $\bm t \sim U(\Delta^{M-1})$.
To do it, the following proposition is useful.
\begin{prop}\label{thm:integ_formula}
Suppose $\sum_{i=1}^M q_i = Q \in \N$ and $r \in \R$, then the following integral formula is satisfied.
\begin{align}
I_{\bm q} (r)
&:=
\paren{
\prod_{i=1}^M
\int_0^\infty
 dt_i t_i^{q_i}
}
\delta \paren{
r - \sum_{i=1}^M t_i
}
= %%%%
r^{Q+M-1} \frac{Q!}{(Q+M-1)! } %\prod_{i=1}^M q_i !
\binom{Q}{\bm q}^{-1}
,
\end{align}
where $\delta (r)$ is the Dirac's delta.
\end{prop}
%
\begin{proof}
One of the easiest proofs is done by making the use of the Laplace transform $\mathcal{L}$ and its inverse $\mathcal{L}^{-1}$, i.e. the nature of $I_{\bm q}(r) = \mathcal{L}^{-1} \sqbra{\mathcal{L} \sqbra{I_{\bm q}}} (r)$.
First, the Laplace transform of $I_{\bm q}$ is
\begin{align}
\mathcal{L} [ I_{\bm q} ] (z)
=
\int_0^\infty dr \ e^{-zr} I_{\bm q} (r)
&=
\Big(
\prod_{i=1}^M
\int_0^\infty
 dt_i t_i^{q_i}
 e^{-zt_i}
\Big)
\notag \\
&=
\frac{Q!}{z^{Q+M}}
\binom{Q}{\bm q}^{-1}.
\end{align}
The inverse Laplace transform of a function $f(z)$ is defined by picking up the residue, the coefficient of $1/z$, of $e^{-zr} f(z)$.
One can calculate the residue of $e^{-zr} \mathcal{L} [ I_{\bm q} ] (z)$ by expanding $e^{-zr}$ with respect to $z$.
It provides $z^{Q+M-1}r^{Q+M-1}/{(Q+M-1)!}$, so we get
\[
\mathcal{L}^{-1} [\mathcal{L} [ I_{\bm q} ]] (r)
=
r^{Q+M-1} \frac{Q!}{(Q+M-1)! } %\prod_{i=1}^M q_i !
\binom{Q}{\bm q}^{-1}.
\]
\end{proof}
%%%%%
Now, let us get back to the proof of the theorem.
The expectation value $\E_{\bm t}[ f(\bm t) ]$ can be represented by the following integral 
\[
\E_{\bm t} [f(\bm t)]
=
(M-1)!
\Big(
\prod_{i=1}^M
\int_0^\infty
 dt_i 
\Big)
\delta \Big(
1 - \sum_{i=1}^M t_i
\Big)
f(\bm t)
\]
The multiplication of $(M-1)!$ is necessary because we need $\E_{\bm t}[1] = 1$.
This is easily checked by using the above proposition by taking $q_i =0$ for all $i$ and $r=1$.
We can also calculate the value $\E_{\bm t}[ \bm t^{\bm d_A + \bm d_B} ]$ by using the proposition with $r=1$ as
\[
\E_{\bm t}[ \bm t^{\bm d_A + \bm d_B} ]
=
\frac{(2D)!(M-1)!}{(2D + M -1)!}
\binom{2D}{\bm d_A + \bm d_B}^{-1}.
\]
By substituting it to \cref{eqn:RN_in_supp}, we get what we want
\begin{align}
R_N
=
\sum_{\bm d_A, \bm d_B \in \N_D^M}
\frac{(2D)!(M-1)!}{(2D + M -1)!}
\binom{D}{\bm d_A}
\binom{D}{\bm d_B}
\binom{2D}{\bm d_A + \bm d_B}^{-1}
\E_{S_N}
\Bigg[
(
\bm P \bm P^\top
)_{\bm d_A \bm d_B}
\Bigg].
\end{align}

\section{Derivation of \texorpdfstring{$\sum_{i, j = 1}^N (A^{-1} \circ A)_{ij} = N$}{sum of Hadamard products}}\label{sec:derivation-of-sum}
\begin{prop}
Let $A = (a_{ij})$ be a regular matrix of size $N$.
Then
\[
    \sum_{i, j = 1}^N (A^{-1} \circ A)_{ij} = N,
\]
where $A \circ B$ is Hadamard product of $A$ and $B$.
\end{prop}
\begin{proof}
Let $\tilde{A}$ be a cofactor matrix and $C_{ij}$ be a $(i, j)$ minor of $A$.
\begin{equation*}
\begin{split}
\sum_{i, j = 1}^N (A^{-1} \circ A)_{ij}  
    &= \sum_{i, j = 1}^N \left(\frac{1}{\det(A)}\tilde{A} \circ A\right)_{ij} \\
    &= \frac{1}{\det(A)}\sum_{i, j = 1}^N (\tilde{A}\circ A)_{ij} \\
    &= \frac{1}{\det(A)}\sum_{i, j = 1}^N a_{ij}C_{ij} \\
    &= \frac{1}{\det(A)}\sum_{i = 1}^N \sum_{j = 1}^N a_{ij}C_{ij}\\
    &= \frac{1}{\det(A)}\sum_{i = 1}^N \det(A) \\
    &= N
\end{split}
\end{equation*}
\end{proof}

\section{Proof of \texorpdfstring{\cref{main:thm:sigma-is-non-singular}}{Theorem 3}}\label{sec:proof-of-regularity}
Put
\begin{align}
\ZZ_{d_A d_B}
:=
\frac{(2D)!(M - 1)!}{(2D + M - 1)!}
 \begin{pmatrix}
  D \\
  \bm{d}_A
 \end{pmatrix}
 \begin{pmatrix}
  D \\
  {\bm d}_B
 \end{pmatrix}
 \begin{pmatrix}
  2D \\
  {\bm d}_A + {\bm d}_B
 \end{pmatrix}^{-1}.
\end{align}
We prove $\ZZ = (\ZZ_{AB})$ is a regular matrix by considering the representation of appropriate spaces.
We start to give some review on bilinear form.
\begin{defn}
Let $V$ be a real vector space. A map $B\colon V\times V \rightarrow \R$ is said to be bininear form if for any $\bm{u}, \bm{v}, \bm{w} \in V$ and $\lambda \in \R$,
\begin{eqnarray*}
\bullet B(\bm{u}+ \bm{v}, \bm{w}) = B(\bm{u}+\bm{w}) + B(\bm{v}, \bm{w})\ \text{and}\  B(\lambda \bm{u}, \bm{v}) = \lambda B(\bm{u}, \bm{v}) \\
\bullet B(\bm{u}, \bm{v}+\bm{w})  = B(\bm{u}+\bm{v}) + B(\bm{u}, \bm{w})\ \text{and}\  B( \bm{u},\lambda \bm{v}) = \lambda B(\bm{u}, \bm{v}) 
\end{eqnarray*}
holds. Moreover, a bilinear form $B$ is called symmetric if $B(\bm{u}, \bm{v})=B(\bm{v}, \bm{u})$ holds for any $\bm{u}, \bm{v} \in V$

\end{defn}

\begin{thm}
Let $V$ be an $n$-dimensional real vector space and $\set{\bm{e}_1, \dots, \bm{e}_n}$ be a basis of $V$. Let  $B\colon V\times V \rightarrow \R$ be a bilinear form. Put $A = (B(\bm{e}_i, \bm{e_j}))$. Then the map 

\[
\begin{array}{ccc}
B':V \times V  & \longrightarrow & \R\\
\ \ \ \ \  \rotatebox{90}{$\in$}    &                 & \rotatebox{90}{$\in$} \\
\ \ \ \ \ (\bm{u}, \bm{v})                   & \longmapsto     & \bm{x}^{\top} A\bm{y}
\end{array},
\]
is a bilinear form which is equal to $B$, where $\bm{u}=\sum_{i=1}^n x_i e_i$ and $\bm{v}=\sum_{i=1}^n y_i e_i$.
\end{thm}
We call the matrix $A$ the representation matrix of $B$ with respect to a basis  $\set{\bm{e}_1, \dots, \bm{e}_n}$ .  By the construction of $A$, we have the following proposition.
\begin{prop}
Let $V$ be an $n$-dimensional real vector space and   $B\colon V\times V \rightarrow \R$ be a bilinear form. Then the representation matrix of $B$ with respect to a basis of $V$ is a symmetric matrix if and only if $B$ is a symmetric bilinear form.
\end{prop} 
\begin{thm}\label{thm:Sigma-is-regular}
$\ZZ$ is a regular matrix.
\end{thm}
\begin{proof}
Let $V_{M, D}$ is the vector space of the homogenous polynomials of degree $D$ with $M$ variables $\bm{x}$.
We define the symmetric bilinear form  by
\[
\begin{array}{ccc}
L:V_{M, D} \times V_{M, D} & \longrightarrow & \R\\
\ \ \ \  \rotatebox{90}{$\in$}    &                 & \rotatebox{90}{$\in$} \\
\ \ \ \  (P, Q)                   & \longmapsto     & \int_{\Delta} P(\bm{x}) Q(\bm{x}) d\bm{x}
\end{array},
\]
where $\Delta = \Set{\bm{x} = (x_1, \dots, x_M) \in \R^M | x_i >0,  x_1 + \dots + x_M = 1}$.

$L$ is clearly symmetric and bilinear.
Let 
\[
    Z = \Set{\frac{\bm{x}^{\bm{d_A}}}{\bm{d_A}!} | \bm{d_A} = (d_A^{(1)}, \dots, d_A^{(M)}) \in \N^M, d_A^{(1)} + \dots + d_A^{(M)} = D}
\]
be a set of polynomials of degree $D$ with $M$ variables.
Then $Z$ is a basis of $V_{M, D}$.
We claim that the representation matrix $G$ of the symmetric bilinear form $L$ with respect to $Z$ is equal to $\ZZ$ up to scalar multiplication.

\begin{claim}\label{thm:rep}
Let $\ZZ$, $Z$, $L$ as above.
Then the representation matrix of $L$ with respect to $Z$ is equal to $r\cdot \ZZ$ for some $r \neq 0 \in \R$.
\end{claim}
\begin{proof}[Proof of \cref{thm:rep}]
We have
\begin{equation*}
\begin{split}
L\paren{\frac{\bm{x}^{\bm{d_A}}}{\bm{d_A}!}, \frac{\bm{x}^{\bm{d_B}}}{\bm{d_B}!}}
  & = \frac{1}{\bm{d_A}!\bm{d_B}!}\int_{\Delta} \bm{x}^{\bm{d_A} + \bm{d_B}}d\bm{x}\\
  & = \frac{1}{\bm{d_A}!\bm{d_B}!} \frac{\Gamma(d_A^{(1)} + d_B^{(1)}) \dots \Gamma(d_A^{(M)} + d_B^{(M)}) }{\Gamma(2D + M)} \\
  & = \frac{(\bm{d_A} + \bm{d_B})!}{\bm{d_A}!\bm{d_B}!}\frac{1}{(2D + M - 1)!}\\
  & = \frac{1}{(D!)^2(M - 1)!} \frac{D!}{\bm{d_A}!} \frac{D!}{\bm{d_B}!} \paren{\frac{2D!}{(\bm{d_A + \bm{d_B}})!}}^{-1} \frac{(2D)!(M - 1)!}{(2D + M - 1)!} \\
  & = \frac{1}{(D!)^2(M - 1)!} \ZZ_{\bm{d_A}\bm{d_B}}.
\end{split}
\end{equation*}
Hence, the claim holds.
\end{proof}
Since $G$ is a symmetric matrix, $G$ is diagonalizable.
Hence, for some basis $Z'$, the representation matrix $H$ of the symmetric bilinear form $L$ with respect to $Z'$ is a diagonal matrix.
In this case, the diagonal components of $H$ is equal to the eigenvalues of $G$ and is equal to $L(P, P)$ for some $P \in Z'$.
Hence, for proving our theorem, it is enough to show that $L(P, P)$ is positive.
To see the positivity of $L(P, P)$, we claim the following.

\begin{claim}\label{thm:hom}
Let $P(\bm{x})$ be a homogenous polynomial of degree $D$ with $M$ variables.
Put $\Delta = \Set{\bm{x} = (x_1, \dots, x_M) \in \R^M | x_i > 0,\ x_1 + \dots + x_M = 1}$.
Assume $\int_{\Delta} P(\bm{x})^2 d \bm{x} = 0$.
Then $P(\bm{x}) = 0$.
\end{claim}

\begin{proof}[Proof of \cref{thm:hom}]
Assume $P(\bm{x}) \neq 0$.
Since $\int_{\Delta} P(\bm{x})^2 d \bm{x} = 0$, we see $P(\bm{x}) = 0$ for any $\bm{x} \in \Delta$.
Since $P(\bm{x}) \neq 0$, we have a point $\bm{a} = (a_1, \dots, a_M) \in \R^M_+$ such that $P(\bm{a}) \neq 0$.
Put $\lambda = \sum a_i$. Then, since $P(\bm{x})$ is a homogeneous polynomial of degree $D$, we have
\[
    P(\bm{a}) = \lambda^D P(\frac{1}{\lambda} \bm{a}) \neq 0.
\]
The non equality holds because $\sum 1 / \lambda \cdot a_i \in \Delta$.
\end{proof}

Since $P$ is a nonzero polynomial, we have
\[
    (P, P) = \int_{\Delta} P(\bm{x})^2 d \bm{x} > 0
\]
by \cref{thm:hom}.
Therefore, any eigenvalue of $G$ is positive.
This implies that $\ZZ$ is a regular matrix.
\end{proof}
%%%%%%
\section{Complete derivation of the risk asymptotics in inductive skeleton fitting}\label{sec:risk-derivation}
Let us begin with how control points on $(m-1)$-subsimplices from minimizing the OLS error
\begin{align}
\frac{1}{N^{(m)}}
\sum_{n=1}^{N^{(m)}}
\norm{
\bm x_n^{(m)}
-
\hat{\bm b} (\bm t_n^{(m)})
}^2
.
\end{align}
In this paper, we assume that sample $\bm x_n$ always represented by sum of a certain B\'esier simplex and noise:
\begin{align}
\bm x_n^{(m)}
=
\bm b (\bm t_n^{(m)})
+
\bm \epsilon_n^{(m)},
\quad
\epsilon_n^{(m)} \sim \mathcal{N}(\bm 0, \sigma^2 \bm I_L)
\end{align}
%%
\paragraph{all-at-once fitting}
All-at-once fitting in our paper can be regarded $m=M$ case.
In this case, we omit the superscript $(m=M)$ and consider minimization of
\begin{align}
\frac{1}{N}
\sum_{n=1}^{N}
\norm{
\bm x_n
-
\hat{\bm b} (\bm t_n)
}^2
&=
\frac{1}{N}
\sum_{n=1}^{N}
\Big| \Big|
{\bm b} (\bm t_n)
-
\hat{\bm b} (\bm t_n)
+
\bm \epsilon_n
\Big| \Big|^2
\notag \\
&=
\frac{1}{N}
\sum_{n=1}^{N}
\Big| \Big|
\sum_{\bm d \in \N_D^M}
\binom{D}{\bm d} \bm t^{\bm d}_n
\bm p_{\bm d}'
+
\bm \epsilon_n
\Big| \Big|^2
\notag \\
&=
\frac{1}{N}
\norm{
\bm Z \bm P
+
\bm Y
}_\mathrm{F}^2,
\end{align}
where the $\norm{\cdot}_\mathrm{F}$ means Frobenius norm of the matrix, and the optimum is determined as
$\bm P_\mathrm{OLS} = - (\bm Z^\top \bm Z)^{-1} \bm Z^\top \bm Y$ by using usual argument.
%%
\paragraph{inductive skeleton fitting}
On the other hand, the above argument should be modified when we consider the inductive skeleton fitting.
First of all, we can reduce the problem to minimization of
\begin{align}
\frac{1}{N^{(m)}}
\sum_{n=1}^{N^{(m)}}
\Big| \Big|
\sum_{\bm d \in \N_D^M}
\binom{D}{\bm d} (\bm t_n^{(m)})^{\bm d}
\bm p_{\bm d}'
+
\bm \epsilon_n^{(m)}
\Big| \Big|^2.
\label{eqn:ols_inductive_supp}
\end{align}
Note that $\bm t_n^{(m)}$ is on $(m-1)$-subsimplices.
It means that the vector
\begin{align}
\bm t_n^{(m)}
=
[
(\bm t_n)_1,
(\bm t_n)_2,
\dots ,
(\bm t_n)_M
]
\end{align}
has $(M-m)$ zero components.
For example, if it is on a subsimplex labelled by
\begin{align}
\bm I = [1, 1, \dots, 1, 0, 0, \dots, 0],
\end{align}
with $m$ ones and $(M-m)$ zeros, then $\bm t_n^{(m)}$ takes
\begin{align}
\bm t_n^{(m)}
=
[
\text{non-zero},
\text{non-zero},
\dots ,
\text{non-zero},
0,
0,
\dots,
0
].
\end{align}
As a result, there is no contribution from $\bm p_{\bm d}'$ which is not on the simplex because such point is labelled by
\begin{align}
\bm d
=
[
d_1,
d_2,
\dots,
d_m,
\text{non-zero},
\text{non-zero},
\dots,
\text{non-zero}
]
\end{align}
because
\begin{align}
(t_n^{(m)})^{\bm d}
=
(\text{non-zero})_1^{d_1}
(\text{non-zero})_2^{d_2}
\dots
(\text{non-zero})_m^{d_m}
(0)^{\text{non-zero}}
(0)^{\text{non-zero}}
\dots
(0)^{\text{non-zero}}
\end{align}
vanishes inside the summation of \cref{eqn:ols_inductive_supp}.
In summary, we should restrict range of the summation in \cref{eqn:ols_inductive_supp} as
\begin{align}
&
\frac{1}{N^{(m)}}
\sum_{n=1}^{N^{(m)}}
\Big| \Big|
\sum_{\bm d \in (\N_D^M)^{(m)}}
\binom{D}{\bm d} (\bm t_n^{(m)})^{\bm d}
\bm p_{\bm d}'
+
\sum_{k < m}
\sum_{\bm d \in (\N_D^M)^{(k)}}
\binom{D}{\bm d} (\bm t_n^{(m)})^{\bm d}
\bm p_{\bm d}'
+
\bm \epsilon_n^{(m)}
\Big| \Big|^2
\notag \\
&=
\frac{1}{N^{(m)}}
\norm{
\bm Z^{(m)} \bm P^{(m)}
+
\sum_{k<m}
\bm Z^{(m)[k]} \bm P^{(k)}
+
\bm Y^{(m)}
}^2_\mathrm{F}.
\label{eqn:ols_ind_supp2}
\end{align}
In the inductive skeleton fitting, we regard all $\bm P^{(k<m)}$ are already determined and fixed in the optimization process of $\bm P^{(m)}$, so the last 2 terms in \cref{eqn:ols_ind_supp2} play a role of $\bm Y$ in all-at-once fitting, and the optimal $\bm P^{(m)}$ is determined as
\begin{align}
{\bm P}_\mathrm{OLS}^{(m)}
= %%%%
-
[({\bm Z}^{(m)})^\top {\bm Z}^{(m)}]^{-1}
({\bm Z}^{(m)})^\top 
\Big(
{\bm Y}^{(m)}
+
\sum_{k<m}
{\bm Z}^{(m) [k]}
{\bm P}_\mathrm{OLS}^{(k)}
\Big).
\label{eqn:rec_supp}
\end{align}

To derive the asymptotics of the risk, we have to consider $\E_\text{samples} [ \bm P \bm P^\top ]$.
In the all-at-once fitting, it is simple as explained in the main body of the paper.
The key idea is calculating the expectation over the noise $\bm Y$ first and making use of the law of large numbers.
On the inductive skeleton fitting, we take same strategy.
%
Let us call the control point matrix $\bm P$ determined by the inductive skeleton fitting as $\bm P_\text{ISK}$.
It decomposes to ``direct sum'' along row of the matrix as
\begin{align}
\bm P_\text{ISK}
=
\begin{pmatrix}
{\bm P}^{(1)}_\mathrm{OLS}
\\
{\bm P}^{(2) }_\mathrm{OLS}
\\
\vdots
\\
{\bm P}^{(M)}_\mathrm{OLS}
\end{pmatrix}
.
\end{align}
If the degree of the B\'ezier simplex is less than $m$, there may be no control point on $(m+k)$-subsimplices.
In such case, we regard $\bm P_\mathrm{OLS}^{(m+k)} = \bm 0$.
By using the decomposition, we can get the ``direct sum'' of the matrix $\bm P_\text{ISK} \bm P_\text{ISK}^\top$ as follows.
\begin{align}
\bm P_\text{ISK} \bm P_\text{ISK}^\top
=
\begin{pmatrix}
{\bm P}^{(1)}_\mathrm{OLS}
({\bm P}^{(1)}_\mathrm{OLS})^\top
&
{\bm P}^{(1)}_\mathrm{OLS}
({\bm P}^{(2)}_\mathrm{OLS})^\top
&
\cdots
&
{\bm P}^{(1)}_\mathrm{OLS}
({\bm P}^{(M)}_\mathrm{OLS})^\top
\\
{\bm P}^{(2)}_\mathrm{OLS}
({\bm P}^{(1)}_\mathrm{OLS})^\top
&
{\bm P}^{(2)}_\mathrm{OLS}
({\bm P}^{(2)}_\mathrm{OLS})^\top
&
\cdots
&
{\bm P}^{(2)}_\mathrm{OLS}
({\bm P}^{(M)}_\mathrm{OLS})^\top
\\
\vdots & \vdots & \ & \vdots
\\
{\bm P}^{(M)}_\mathrm{OLS}
({\bm P}^{(1)}_\mathrm{OLS})^\top
&
{\bm P}^{(M)}_\mathrm{OLS}
({\bm P}^{(2)}_\mathrm{OLS})^\top
&
\cdots
&
{\bm P}^{(M)}_\mathrm{OLS}
({\bm P}^{(M)}_\mathrm{OLS})^\top
\end{pmatrix}
.
\end{align}
Therefore, to get the asymptotic form of the risk, it is sufficient to consider $\E_\text{samples}[{\bm P}^{(i)}_\mathrm{OLS}
({\bm P}^{(j)}_\mathrm{OLS})^\top]$ for $i, j = 1, 2, \dots, M$.
To get a grasp of how it calculated, let us take $(i,j) = (1,1)$ and $(i, j) =  (2, 1)$ here.
First, $(i,j) = (1,1)$ is the simplest case because there is no lower-dimensional subsimplex.
So we do not need 2nd term in \cref{eqn:rec_supp} and get
\begin{align}
&\E_\text{samples}
\Big[
\bm P_\mathrm{OLS}^{(1)}
(\bm P_\mathrm{OLS}^{(1)})^\top
\Big]
\notag \\
&=
\E_\text{samples}
\Bigg[
\Big[
-
[({\bm Z}^{(1)})^\top {\bm Z}^{(1)}]^{-1}
({\bm Z}^{(1)})^\top 
{\bm Y}^{(1)}
\Big]
\Big[
-
({\bm Y}^{(1)})^\top
{\bm Z}^{(1)}
[({\bm Z}^{(1)})^\top {\bm Z}^{(1)}]^{-1}
\Big]
\Bigg]
\notag \\
&=
\E_{\text{samples}}
\Bigg[
[({\bm Z}^{(1)})^\top {\bm Z}^{(1)}]^{-1}
({\bm Z}^{(1)})^\top 
\E_{\bm Y^{(1)}}
\Big[
{\bm Y}^{(1)}
({\bm Y}^{(1)})^\top
\Big]
{\bm Z}^{(1)}
[({\bm Z}^{(1)})^\top {\bm Z}^{(1)}]^{-1}
\Bigg]
\notag \\
&=
\E_{\text{samples}}
\Bigg[
[({\bm Z}^{(1)})^\top {\bm Z}^{(1)}]^{-1}
({\bm Z}^{(1)})^\top 
\Big[
\sigma^2 L 1_{N^{(1)}}
\Big]
{\bm Z}^{(1)}
[({\bm Z}^{(1)})^\top {\bm Z}^{(1)}]^{-1}
\Bigg]
\notag \\
&=
\sigma^2 L
\E_{\text{samples}}
\Bigg[
[({\bm Z}^{(1)})^\top {\bm Z}^{(1)}]^{-1}
\Bigg].
\label{eqn:P1P1}
\end{align}
In fact, this simplification occurs in calculation with all-at-once fitting.
So $(i, j) = (1,1)$ case is in same situation of the calculation of the asymptotic risk of the all-at-once fitting.
How about $(i, j) = (2, 1)$ ?
This is also simple because we can reduce the calculation to lower-dimensional object by using \cref{eqn:rec_supp}:
\begin{align}
&\E_\text{samples}
\Big[
\bm P_\mathrm{OLS}^{(2)}
(\bm P_\mathrm{OLS}^{(1)})^\top
\Big]
\notag \\
&=
\E_\text{samples}
\Bigg[
\Big[
-
[({\bm Z}^{(2)})^\top {\bm Z}^{(2)}]^{-1}
({\bm Z}^{(2)})^\top 
\Big(
{\bm Y}^{(2)}
+
\bm Z^{(2)[1]}
\bm P_\mathrm{OLS}^{(1)}
\Big)
\Big]
(\bm P_\mathrm{OLS}^{(1)})^\top
\Bigg]
\notag \\
&=
\E_\text{samples}
\Bigg[
\Big[
-
[({\bm Z}^{(2)})^\top {\bm Z}^{(2)}]^{-1}
({\bm Z}^{(2)})^\top 
\Big(
\E_{\bm Y^{(2)}}
[
{\bm Y}^{(2)}
]
+
\bm Z^{(2)[1]}
\bm P_\mathrm{OLS}^{(1)}
\Big)
\Big]
(\bm P_\mathrm{OLS}^{(1)})^\top
\Bigg]
\notag \\
&=
\E_\text{samples}
\Bigg[
\Big[
-
[({\bm Z}^{(2)})^\top {\bm Z}^{(2)}]^{-1}
({\bm Z}^{(2)})^\top 
\Big(
[
0
]
+
\bm Z^{(2)[1]}
\bm P_\mathrm{OLS}^{(1)}
\Big)
\Big]
(\bm P_\mathrm{OLS}^{(1)})^\top
\Bigg]
\notag \\
&=
- \sigma^2 L
\E_\text{samples}
\Bigg[
[({\bm Z}^{(2)})^\top {\bm Z}^{(2)}]^{-1}
[({\bm Z}^{(2)})^\top \bm Z^{(2)[1]}]
[({\bm Z}^{(1)})^\top {\bm Z}^{(1)}]^{-1}
\Bigg].
\label{eqn:P2P1}
\end{align}
With generic $(i, j)$, we can perform same procedure.

Now, let us make next step.
To calculate the asymptotic form, it is important to notice that the following explicit forms are averages on sampling from $\bm t_n^{(m)} \sim U(\cup_{|\bm I|=m} \Delta^{\bm I})$.
\begin{align}
&
\frac{1}{N^{(m)}}
\Big(
(\bm Z^{(m)})^\top
\bm Z^{(m)[k]}
\Big)_{\bm d_m \bm d_k}
=
\binom{D}{\bm d_m}
\binom{D}{\bm d_k}
\sum_{n=1}^{N^{(m)}}
\frac{1}{N^{(m)}}
(\bm t_n^{(m)})^{\bm d_m + \bm d_k}
\label{eqn:Zmk}
\\
&
\frac{1}{N^{(m)}}
\Big(
(\bm Z^{(m)})^\top
\bm Z^{(m)}
\Big)_{\bm d_A \bm d_B}
=
\binom{D}{\bm d_A}
\binom{D}{\bm d_B}
\sum_{n=1}^{N^{(m)}}
\frac{1}{N^{(m)}}
(\bm t_n^{(m)})^{\bm d_A + \bm d_B}
\label{eqn:Zmm}
\end{align}
By applying the law of large numbers, they converge to exact values of expectation:
\begin{align}
&
\cref{eqn:Zmk}
\to
\binom{D}{\bm d_m}
\binom{D}{\bm d_k}
\E_{\bm t^{(m)} \sim U(\cup_{|\bm I|=m} \Delta^{\bm I})} [
(\bm t^{(m)})^{\bm d_m + \bm d_k}
]
\label{eqn:Lmk}
\\
&
\cref{eqn:Zmm}
\to
\binom{D}{\bm d_A}
\binom{D}{\bm d_B}
\E_{\bm t^{(m)} \sim U(\cup_{|\bm I|=m} \Delta^{\bm I})} [
(\bm t^{(m)})^{\bm d_A + \bm d_B}
]
\label{eqn:Lmm}
\end{align}
To compute expectation values over $U(\cup_{|\bm I|=m} \Delta^{\bm I})$, the next proposition is useful.
\begin{prop}
\begin{align}
\E_{\bm t \sim U(\cup_{|\bm I|=m} \Delta^{\bm I})} [
\bm t^{\bm d^{(m)} + \bm d^{(k)}}
]
=
\frac{(m-1)!}{\ _M C_m}
\sum_{|\bm I| = m}
\frac{
1_{{\bm I}=(\bm d ^{(m)} +  \bm d ^{(k)} )_{01} }
\prod_{I_i = 1}( \bm d ^{(m)} +  \bm d ^{(k)}  )_i !}{
[\sum_{I_i = 1} ( \bm d ^{(m)} +  \bm d ^{(k)}  )_i 
+ m - 1]!
},
\end{align}
where $\bm I = [I_1, I_2, \dots, I_M]$ represents vector with binary components, $I_i \in \set{0, 1}$ and $\card{\bm I} = \sum_{i=1}^M I_i$.
\end{prop}
%%%
\begin{proof}
The expectation value with the uniform distribution on $(m-1)$-subsimplices should be expressed by integral
\begin{align}
\E_{\bm t \sim U(\cup_{|\bm I|=m} \Delta^{\bm I})} [ f(\bm t) ]
=
(\text{const.})
\int_{0}^1 \Big( 
\prod_{i=1}^M dt_i
\Big)
\ f(\bm t)
\sum_{|\bm I| = m}
\delta \Big(
1 - \sum_{I_i =1} t_i
\Big)
\prod_{I_i =0}
\delta (t_i)
\label{eqn:exp_sub}
\end{align}
because it can be regarded integral over $\Delta^{M}$ restricted to the set of subsimplices.
The normalization constant can be calculated by the integral $f(\bm t) = 1$ inserted.
In this situation, there is no special direction on $\bm t$, and all contributions from the sum $\sum_{|\bm I|=m}$ should be identical.
Therefore, we do not need to calculate all of them but just select one configuration $\bm I$ easy to calculate and multiply the number of combination giving $|\bm I|=m$, i.e. $\ _M C_m$.
Let us take
\begin{align}
\bm I = [\underbrace{1,1, \dots, 1}_{m}, 0, 0, \dots, 0]
\end{align}
then the integral reduces to
\begin{align}
\int_{0}^1 \Big( 
\prod_{i=1}^m dt_i
\Big)
\delta \Big(
1 - \sum_{i=1}^m t_i
\Big)
=
\frac{1}{(m-1)!}
\end{align}
where we use the integral formula derived in \ref{thm:integ_formula}.
It determines the constant in \cref{eqn:exp_sub}, 
\begin{align}
(\text{const.})
=
\frac{(m-1)!}{\ _M C_m}
.
\end{align}
Next task is performing the integral $f(\bm t) = \bm t^{\bm d^{(m)} + \bm d^{(k)}}$ inserted.
It is also not so difficult.
To get nonzero value from $\bm I$-th contribution, $\bm d^{(m)} + \bm d^{(k)}$ should be on subsimplex labelled by $\bm I$.
If not so, the integrand includes
\begin{align}
\int dt \
t^\text{non-zero}
\delta(t)
=
0^\text{non-zero}
=0.
\end{align}
To represent this condition, let us introduce the notation
\begin{align}
(\bm d)_{01}
=
[d^\mathrm{bin}_1, d^\mathrm{bin}_2, \dots, d^\mathrm{bin}_M],
\quad
d^\mathrm{bin}_i
=
\begin{cases}
0 & \text{if $d_i = 0$}\\
1 & \text{otherwise}
\end{cases}
\end{align}
By using this notation, the nonzero contribution condition is represented by the insertion
\begin{align}
1_{\bm I = (\bm d^{(m)} + \bm d^{(k)})_{01} }.
\end{align}
If index $\bm I$ in the summation $\sum_{|\bm I| = m}$ enjoys this condition, the remaining part is integral on an $(m-1)$-subsimplex labelled by $\bm I$, and the contribution is
\begin{align}
\frac{\prod_{I_i=1} (\bm d^{(m)} + \bm d^{(k)})_i ! }{
[
\sum_{I_i=1}(\bm d^{(m)} + \bm d^{(k)})_i + m -1 ]!
}
\end{align}
which is derived in the same way of the integral formula shown in \ref{thm:integ_formula}.
\end{proof}
Applying this proposition to \cref{eqn:Lmk,eqn:Lmm}, we get the asymptotic formula of $
\E_\text{samples}[
\bm P_\text{ISK} \bm P_\text{ISK}^\top]$, or equivalently, $\E_\text{samples}[\bm P_\mathrm{OLS}^{(i)} (\bm P_\mathrm{OLS}^{(j)})^\top]$.
For example,
\begin{align}
\E_\text{samples}[\bm P_\mathrm{OLS}^{(1)} (\bm P_\mathrm{OLS}^{(1)})^\top]
&=
\cref{eqn:P1P1}
=
\frac{\sigma^2 L}{N^{(1)}} \E_\text{samples}
\Big[
[\frac{1}{N^{(1)}} (\bm Z^{(1)})^\top \bm Z^{(1)} ]^{-1}
\Big]
\notag \\
& \to
\frac{\sigma^2 L}{N^{(1)}} \E_\text{samples}
\Big[
\bm \Lambda_{(1)}
\Big]
=
\frac{\sigma^2 L}{N^{(1)}} \bm \Lambda_{(1)},
\end{align}
and
\begin{align}
&\E_\text{samples}[\bm P_\mathrm{OLS}^{(2)} (\bm P_\mathrm{OLS}^{(1)})^\top]
\notag \\
&=
\cref{eqn:P2P1}
=
- \frac{\sigma^2 L}{N^{(1)}}
\E_\text{samples}
\Bigg[
[\frac{1}{N^{(2)}} ({\bm Z}^{(2)})^\top {\bm Z}^{(2)}]^{-1}
[\frac{1}{N^{(2)}}({\bm Z}^{(2)})^\top \bm Z^{(2)[1]}]
[\frac{1}{N^{(1)}}({\bm Z}^{(1)})^\top {\bm Z}^{(1)}]^{-1}
\Bigg]
\notag \\
& \to
- \frac{\sigma^2 L}{N^{(1)}}
\E_\text{samples}
\Bigg[
\bm \Lambda_{(2)}
\bm \Lambda^{(2)[1]}
\bm \Lambda_{(1)}
\Bigg]
=
- \frac{\sigma^2 L}{N^{(1)}}
\bm \Lambda_{(2)}
\bm \Lambda^{(2)[1]}
\bm \Lambda_{(1)},
\end{align}
where $\bm \Lambda$s are defined in the main body of the paper.
As one can see, to complete our argument, it is sufficient to show the next proposition:
\begin{prop}
\begin{align}
&\E_\text{all noises}
\Big[
\bm P^{(i)}_\mathrm{OLS}
(\bm P^{(j)}_\mathrm{OLS})^\top
\Big]
\notag \\
&=
\sigma^2 L
\sum_{
\substack{
m \leq i
\\
m \leq j
}}
\sum_{
\substack{
m\leq k_1<\cdots<k_{\heartsuit}< i
\\
m \leq l_1<\cdots<l_{\spadesuit}<j
}
}
\frac{(-1)^{\heartsuit + \spadesuit}}{N^{(m)}}
%\notag \\ \nline \quad \times %\qquad %\qquad
\hat{\bm \Lambda}_{(i)}
\hat{\bm \Lambda}^{(i)[k_{\heartsuit}]}
\hat{\bm \Lambda}_{(k_{\heartsuit} )}
%\hat{\bm \Lambda}^{(k_{\#_i})[k_{\#_i -1}]}
\cdots
\hat{\bm \Lambda}^{(k_1)[m]}
\hat{\bm \Lambda}_{(m)}
\hat{\bm \Lambda}^{[m](l_1)}
\cdots
\hat{\bm \Lambda}_{(l_{\spadesuit} )}
\hat{\bm \Lambda}^{[l_{\spadesuit}](j)}
\hat{\bm \Lambda}_{(j)},
\label{eqn:31}
\end{align}
where
\begin{align}
&\hat{\bm \Lambda}^{(m)[k]}
=
\frac{1}{N^{(m)}}
({\bm Z}^{(m)})^\top 
{\bm Z}^{(m)[k]}
\\
&
\hat{\bm \Lambda}^{[k](m)}
=
(\hat{\bm \Lambda}^{(m)[k]})^\top
\\
&
\hat{\bm \Lambda}_{(m)}
=
[\frac{1}{N^{(m)}}({\bm Z}^{(m)})^\top 
{\bm Z}^{(m)}]^{-1}
\end{align}
\end{prop}
%%%
\begin{proof}
We show it by induction.
For $(i, j) = (1, 1)$, the only possible contribution of the summation in \cref{eqn:31} is $m=1$ and the (RHS) reduces to
\begin{align}
\sigma^2 L
\frac{(-1)^{0+0}}{N^{(1)}}
\hat{\bm \Lambda}_{(1)}
\hat{\bm \Lambda}^{(1)[1]}
\hat{\bm \Lambda}_{(1)}
\hat{\bm \Lambda}^{[1](1)}
\hat{\bm \Lambda}_{(1)}
=
\frac{\sigma^2 L}{N^{(1)}}
\hat{\bm \Lambda}_{(1)},
\end{align}
which is satisfied as we shown already.
Now let us assume \cref{eqn:31} with $(k, j)$ for all $k< i+1$, then, by using the recursive relation \cref{eqn:rec_supp}, we get
\begin{align}
&\E_\text{all noises}
\Big[
\bm P_\mathrm{OLS}^{(i+1)}
(\bm P_\mathrm{OLS}^{(j)})^\top
\Big]
\notag \\
&=
\E_\text{all noises}
\Big[
\Big\{
-
[(\bm Z^{(i+1)})^\top \bm Z^{(i+1)} ]^{-1}
(\bm Z^{(i+1)})^\top
\Big(
\bm Y^{(i+1)}
+
\sum_{k<i+1}
\bm Z^{(i+1)[k]}
\bm P_\mathrm{OLS}^{(k)}
\Big)
\Big\}
(\bm P_\mathrm{OLS}^{(j)})^\top
\Big].
\label{eqn:41}
\end{align}
1st term gives
\begin{align}
&\E_\text{all noises}
\Big[
-
[(\bm Z^{(i+1)})^\top \bm Z^{(i+1)} ]^{-1}
(\bm Z^{(i+1)})^\top
\bm Y^{(i+1)}
(\bm P_\mathrm{OLS}^{(j)})^\top
\Bigg]
\notag \\
&= %%%%
\E_\text{all noises}
\Big[
-
[(\bm Z^{(i+1)})^\top \bm Z^{(i+1)} ]^{-1}
(\bm Z^{(i+1)})^\top
\bm Y^{(i+1)}
\Big\{
-
(\bm Y^{(j)})^\top
\notag \\ & \hspace{180pt}
-
\sum_{m < j}
(\bm P_\mathrm{OLS}^{(m)})^\top
\bm Z^{[m](j)}
\Big\}
\bm Z^{(j)}
[(\bm Z^{(j)})^\top \bm Z^{(j)} ]^{-1}
\Big]
\notag \\
&= %%%%
\E_\text{all noises}
\Big[
\underbrace{
[(\bm Z^{(i+1)})^\top \bm Z^{(i+1)} ]^{-1}
(\bm Z^{(i+1)})^\top
\bm Y^{(i+1)}
(\bm Y^{(j)})^\top
\bm Z^{(j)}
[(\bm Z^{(j)})^\top \bm Z^{(j)} ]^{-1}
}_{(*1)}
\notag \\ &
+
\underbrace{
[(\bm Z^{(i+1)})^\top \bm Z^{(i+1)} ]^{-1}
(\bm Z^{(i+1)})^\top
\bm Y^{(i+1)}
\sum_{m < j}
(\bm P_\mathrm{OLS}^{(m)})^\top
\bm Z^{[m](j)}
\bm Z^{(j)}
[(\bm Z^{(j)})^\top \bm Z^{(j)} ]^{-1}
}_{(*2)}
\Big].
\end{align}
If $i+1 = j$, the first term only contribute because there is no $(\bm Y^{(i+1)})^\top$ in the summation $\sum_{m < j=i+1}$ of the second term, and it gives
\begin{align}
\E (*1)
=
[(\bm Z^{(i+1)})^\top \bm Z^{(i+1)} ]^{-1}
=
\frac{1}{N^{(i+1)}}
\hat{\bm \Lambda}_{(i+1)}
\label{eqn:43}
\end{align}
If $i+1 \neq j$, the first term vanish in the same reason.
In this case, the second term possibly contribute if $i+1 < j$.
By applying the recursive formula \cref{eqn:rec_supp} to $\bm P_\mathrm{OLS}^{(m)}$ repeatedly until $(\bm Y^{(i+1)})^\top$ appears , we get
\begin{align}
\E (*2)
=
\sigma^2 L
\sum_{m < j}
\sum_{m \leq l_1 < \dots < l_\spadesuit <j }
\frac{(-1)^{\spadesuit}}{N^{(i+1)}}
\hat{\bm \Lambda}_{(i+1)}
\hat{\bm \Lambda}^{[i+1](l_1)}
\hat{\bm \Lambda}_{(l_1)}
\dots
\hat{\bm \Lambda}_{(l_\spadesuit)}
\hat{\bm \Lambda}^{[l_\spadesuit](j)}
\hat{\bm \Lambda}_{(j)}.
\label{eqn:44}
\end{align}

On the other hand, the 2nd term in \cref{eqn:41} is
\begin{align}
&
\E_\text{all noises}
\Big[
-
[(\bm Z^{(i+1)})^\top \bm Z^{(i+1)}]^{-1}
(\bm Z ^{(i+1)})^\top
\sum_{k<i+1}
\bm Z^{(i+1)[k]}
\bm P_\mathrm{OLS}^{(k)}
(\bm P_\mathrm{OLS}^{(j)})^\top
\Big]
\notag \\
&=
\sum_{k<i+1}
(-1)
\hat{\bm \Lambda}_{(i+1)}
\hat{\bm \Lambda}^{(i+1)[k]}
\sigma^2 L
\sum_{
\substack{
m \leq k
\\
m \leq j
}}
\sum_{
\substack{
m\leq k_1<\cdots<k_{\heartsuit}<k
\\
m \leq l_1<\cdots<l_{\spadesuit}<j
}
}
\notag \\ & \qquad\qquad\qquad
\frac{(-1)^{\heartsuit + \spadesuit}}{N^{(m)}}
%\notag \\ \nline \quad \times %\qquad %\qquad
\hat{\bm \Lambda}_{(k)}
\hat{\bm \Lambda}^{(k)[k_{\heartsuit}]}
\hat{\bm \Lambda}_{(k_{\heartsuit} )}
%\hat{\bm \Lambda}^{(k_{\#_i})[k_{\#_i -1}]}
\cdots
\hat{\bm \Lambda}^{(k_1)[m]}
\hat{\bm \Lambda}_{(m)}
\hat{\bm \Lambda}^{[m](l_1)}
\cdots
\hat{\bm \Lambda}_{(l_{\spadesuit} )}
\hat{\bm \Lambda}^{[l_{\spadesuit}](j)}
\hat{\bm \Lambda}_{(j)}
\notag \\
&=
\sigma^2 L
\sum_{
\substack{
m \leq k
\\
m \leq j
}}
\sum_{
\substack{
m\leq k_1<\cdots<k_{\heartsuit}<k<i+1
\\
m \leq l_1<\cdots<l_{\spadesuit}<j
}
}
\hat{\bm \Lambda}_{(i+1)}
\hat{\bm \Lambda}^{(i+1)[k]}
\notag \\ & \qquad\qquad\qquad
\frac{(-1)^{\heartsuit+1 + \spadesuit}}{N^{(m)}}
%\notag \\ \nline \quad \times %\qquad %\qquad
\hat{\bm \Lambda}_{(k)}
\hat{\bm \Lambda}^{(k)[k_{\heartsuit}]}
\hat{\bm \Lambda}_{(k_{\heartsuit} )}
%\hat{\bm \Lambda}^{(k_{\#_i})[k_{\#_i -1}]}
\cdots
\hat{\bm \Lambda}^{(k_1)[m]}
\hat{\bm \Lambda}_{(m)}
\hat{\bm \Lambda}^{[m](l_1)}
\cdots
\hat{\bm \Lambda}_{(l_{\spadesuit} )}
\hat{\bm \Lambda}^{[l_{\spadesuit}](j)}
\hat{\bm \Lambda}_{(j)}
\notag \\
&= %%%%
\sigma^2 L
\sum_{
\substack{
m \leq k
\\
m \leq j
}}
\sum_{
\substack{
m\leq k_1<\cdots<k_{\heartsuit}<k<i+1
\\
m \leq l_1<\cdots<l_{\spadesuit}<j
}
}
\notag \\ & 
\frac{(-1)^{\heartsuit+1 + \spadesuit}}{N^{(m)}}
\hat{\bm \Lambda}_{(i+1)}
\hat{\bm \Lambda}^{(i+1)[k]}
%\notag \\ \nline \quad \times %\qquad %\qquad
\hat{\bm \Lambda}_{(k)}
\hat{\bm \Lambda}^{(k)[k_{\heartsuit}]}
\hat{\bm \Lambda}_{(k_{\heartsuit} )}
%\hat{\bm \Lambda}^{(k_{\#_i})[k_{\#_i -1}]}
\cdots
\hat{\bm \Lambda}^{(k_1)[m]}
\hat{\bm \Lambda}_{(m)}
\hat{\bm \Lambda}^{[m](l_1)}
\cdots
\hat{\bm \Lambda}_{(l_{\spadesuit} )}
\hat{\bm \Lambda}^{[l_{\spadesuit}](j)}
\hat{\bm \Lambda}_{(j)}
\end{align}
This is close to what we want to show by renaming $k = k_{\heartsuit + 1}$, but it lacks contribution from $m = i+1$:
\begin{align}
\sigma^2 L
\sum_{m \leq j}
\sum_{m \leq l_1 < \dots < l_\spadesuit <j }
\frac{(-1)^{\spadesuit}}{N^{(i+1)}}
\hat{\bm \Lambda}_{(i+1)}
\hat{\bm \Lambda}^{[i+1](l_1)}
\hat{\bm \Lambda}_{(l_1)}
\dots
\hat{\bm \Lambda}_{(l_\spadesuit)}
\hat{\bm \Lambda}^{[l_\spadesuit](j)}
\hat{\bm \Lambda}_{(j)}
.
\end{align}
But is is compensated by \cref{eqn:43,eqn:44}, if $j=i+1$ and $j \neq i+1$ respectively, so we succeed in deriving the equation with $(i+1, j)$.

On $(i, j+1)$ case, we can show it by repeating the above argument in transposed version.
\end{proof}
Now we complete the derivation of the asymptotic form of $\E_\text{samples}[\bm P_\text{ISK} \bm P_\text{ISK}^\top]$.




%%%%%%  
The following proposition gives the optimal value of the risk of inductive skeleton fitting in the case of $D=2$.
\begin{prop}
Let $N$ be a natural number and $a, b$ be positive real numbers.
Let $f(x) = \frac{a}{x} + \frac{b}{N - x}$.
Assume $a > b$ and $\frac{a - \sqrt{ab}}{a - b} < 1$.
Then $\min \Set{f(x) | x \in \R, 0 < x < N} = \frac{a + b + 2 \sqrt{ab}}{N}$.
\end{prop}

\begin{proof}
At first, we consider the differential of $f(x)$.
We have
\[
    f'(x) = -\frac{(a - b) x^2 - 2aNx + aN^2}{x^2 (N - x)^2}.
\]
$f'(x) = 0$ if and only if $x = \frac{a \pm \sqrt{ab}}{a - b}N$.
By the assumptions, $f(x)$ takes the minimal value at $x = \frac{a - \sqrt{ab}}{a - b}N$.
Hence, we have
\[
    \min \Set{f(x) | x \in \R, 0 < x < N} = f(\frac{a - \sqrt{ab}}{a - b} N) = \frac{a + b + 2 \sqrt{ab}}{N}.
\]
\end{proof}


%How to generate data sets, sample size, etc.
\section{All experiments}\label{sec:numerical-experiments}
This section provides all experiments we conducted and their detailed settings for completeness and reproducibility.

\subsection{Synthetic instances}
First of all, we present detailed experimental settings and results of synthetic instances described in \cref{main:sec:synthetic-instances} of the main paper.

\subsubsection{Data generation}
We consider the following data generating process to create synthetic instances.
Given parameters $(L, M, N)$, we first define true control points $\bm p_{\bm d}\in \R^L~(\bm d \in \N^M_D)$ as follows:
\begin{equation}\label{eqn:def_control_points}
    \bm p_{\bm d} := \sum_{j=1}^M \frac{d_j}{D} \bm e_j,
\end{equation}
where $\bm e_j \in \R^L~(j = 1, \dots, M)$ is a unit vector whose $j$-th element is one and the others are zeros.
The B\'ezier simplex defined by \cref{eqn:def_control_points} is a unit $(M-1)$-simplex on $\R^L$ and their $M$ vertices are $\bm e_j~(j = 1, \dots, M)$.

Next, we randomly generated $N$ training points $\set{(\bm t_n, \bm x_n)}_{n = 1}^N$%$S_N = \Set{(\bm t_1, \bm x_1), \dots, (\bm t_N, \bm x_N)}$ and $S_{N(0)}, S_{N(1)}$
for the all-at-once fitting and the inductive skeleton fitting respectively.
For the all-at-once fitting, we generated parameters $\bm t_n~(n = 1, \dots, N)$ randomly from an uniform distribution $U(\Delta^{M-1})$, and set up $\bm x_n$ as follows:
\begin{equation}\label{eqn:exp_add_noise}
    \bm x_n = \sum_{\bm d \in \N^M_D} \binom{D}{\bm d} \bm t_n^{\bm d} \bm p_{\bm d} + \bm \varepsilon_n \quad (n = 1, \dots, N),
\end{equation}
where $\bm \varepsilon_n~(n = 1, \dots, N)$ is a noise generated from a normal distribution $N(\bm 0, 0.1^2 \bm I)$.

For the inductive skeleton on the other hand, we set up training points for each dimensional subsimplex of $\Delta^{(m-1)} = \cup_{\card{\bm I} = m} \Delta^{\bm{I}}~(m = 1, \dots, M)$ to be learned.
First, we decoupled $N$ into $N^{(m)}~(m = 0, \dots, M-1)$.
To obtain parameters $\bm{t}^{(m)}_n~(n = 1, \dots, N^{(m)})$ from $\Delta^{(m)} = \cup_{\card{I} = m} \Delta^{\bm{I}}$, we further divided $N^{(m)}$ by the number of $m$-subsimplices $\Delta^{\bm{I}}~(\card{I} = m)$, and generated parameters of equal size from each uniform distribution $U(\Delta^{\bm{I}})$.
%generated parameters $\bm{t}^{(m)}_n~(n = 1, \dots, N^{(m)}$ for each subsimplex from an uniform distribution $U(\Delta^m)$.
Then we produced training points $\bm x^{(m)}_n~(n = 1, \dots, N^{(m)})$ in the same way as \cref{eqn:exp_add_noise} and obtained training points $\set{(\bm x_n^{(m)}, \bm t_n^{(m)})}_{n=1}^{N^{(m)}}$ for each $(m)$-skeleton.

%In the experiments,
\subsubsection{Experimental settings}
Experiments were conducted on the following tuple $(N, M, L)$ with $D \in \Set{2, 3}$:
\begin{itemize}
    \item  $(M, L) = (8, 100)$ and $N \in \Set{250, 500, 1000, 2000}$;
    \item  $(N, L) = (1000, 100)$ and $M \in \Set{3, 4, 5, 6, 7, 8}$;
    \item  $(M, N) = (8, 1000)$ and $ M \in \Set{8, 25, 50, 100}$.
\end{itemize}

To verify the asymptotic risks derived in Section 3.1 and Section 3.2, we compared the following three methods:
\begin{description}
    \item[all-at-once] the all-at-once fitting (Section 3.1);
    \item[inductive skeleton (non-optimal)] the inductive skeleton fitting (Section 3.2) with $N^{(0)} = \dots = N^{(M-1)} = N / M$, which is not the minimizer of the risk of the inductive skeleton fitting;
    \item[inductive skeleton (optimal)] the inductive skeleton fitting (Section 3.2) where $N^{(0)}, N^{(1)}, \dots, N^{(M-1)}$ are determined by minimizing the risk of the inductive skeleton fitting under the constraints $\sum_{m = 0}^{M-1} N^{(m)} = N$ and $N^{(m)}\geq 0~(m = 0, \dots, M-1)$. The actual sample sizes $N^{(m)}$ for each $(M,D)$ are described in \cref{tab:optimal-subsample-ratio}.
    \Cref{tab:optimal-subsample-ratio} shows the optimal solutions of $N^{(m)}$ for each pairs $(M, D)$.
\end{description}
%For the inductive skeleton fitting (optimal), we decoupled $N$ into $N(m)$ according to the following table:
\begin{table}[ht]
    \centering
    \caption{Optimal subsample ratio for inductive skeleton fitting ($D$: degree of B\'ezier simplex, $M$: dimension of B\'ezier simplex, $N$: total sample size, $N^{(m)}$: sample size of $m$-skeleton).}
    \label{tab:optimal-subsample-ratio}
    \begin{tabular}{cccccccc}
        \toprule
        $D$     &$N^{(m)}$ &$M = 3$  &$M = 4$  &$M = 5$  &$M = 6$  &$M = 7$  &$M = 8$ \\ \midrule
        $2$     &$N^{(0)}$ &0.262$N$ &0.229$N$ &0.228$N$ &0.233$N$ &0.238$N$ &0.242$N$\\ 
                &$N^{(1)}$ &0.738$N$ &0.771$N$ &0.772$N$ &0.767$N$ &0.762$N$ &0.758$N$\\  \hline
        $3$     &$N^{(0)}$ &0.118$N$ &0.083$N$ &0.066$N$ &0.058$N$ &0.052$N$ &0.051$N$\\
                &$N^{(1)}$ &0.576$N$ &0.444$N$ &0.547$N$ &0.577$N$ &0.589$N$ &0.613$N$\\ 
                &$N^{(2)}$ &0.306$N$ &0.473$N$ &0.387$N$ &0.365$N$ &0.359$N$ &0.336$N$\\ 
        \bottomrule
    \end{tabular}
\end{table}

In this experiment, we estimated the B\'ezier simplex with degree $D = 2$ and 3 respectively.

\subsubsection{All results}\label{sec:results}
\Cref{fig:mse_vsN} shows box plots of MSEs over 20 trials and our theoretical risks for both the all-at-once fitting and the inductive skeleton fitting for each $N \in \Set{250, 500, 1000, 2000}$ with $(L, M) = (100, 8)$ and $D \in \Set{2, 3}$.
We observed that these figures empirically show that our theoretical risks are correct for both $D = 2$ and 3, and the gap between the actual MSEs and the risks are sufficiently small at $N = 1000$.
For both $D = 2$ and 3, the inductive skeleton (optimal) always achieved lower MSEs than that of the inductive skeleton (non-optimal).
This result suggests the efficiency of optimizing the risk of the inductive skeleton fitting with respect to the sample sizes $N^{(m)}$ of each dimension.
In addition, the inductive skeleton fitting (optimal) also outperformed the all-at-once fitting in the case of $D = 2$.
This result supports the discussion described in Section 3.3.
\begin{figure}[ht]
 \begin{minipage}{0.49\hsize}
        \centering
    \includegraphics[width=1\textwidth]{neurips2019/fig/D=2_M=8_L=100.pdf}
    \subcaption{D=2}
    \label{fig:mse_vsN_d2}
 \end{minipage}
 \begin{minipage}{0.49\hsize}
        \centering
    \includegraphics[width=1\textwidth]{neurips2019/fig/D=3_M=8_L=100.pdf}
    \subcaption{D=3}
    \label{fig:mse_vsN_d3}
 \end{minipage}
 \caption{Sample size $N$ vs. MSE with $(L,M)=(0.1,100,8)$ (boxplots over 20 trials and theoretical risks).}
 \label{fig:mse_vsN}
\end{figure}


\Cref{fig:mse_vsM} shows box plots of MSEs over 20 trials and our theoretical risks for each $M \in \Set{3, 4, 5, 6, 7, 8}$ with $(L, N) = (100, 1000)$.
As well as \cref{fig:mse_vsN}, the inductive skeleton fitting always outperformed the all-at-once fitting in the case of $D = 2$.
Furthermore, the difference of MSEs between the inductive skeleton fitting and the all-at-once fitting gets wider as $M$ grows.
This suggests that the inductive skeleton fitting with $D = 2$ approximates more effectively than the all-at-once fitting does for a high-dimensional $M$.
\begin{figure}[ht]
 \begin{minipage}{0.49\hsize}
        \centering
    \includegraphics[width=1\textwidth]{neurips2019/fig/D=2_L=100_N=1000.pdf}
    \subcaption{$D = 2$}    \label{fig:mse_vsM_d2}
 \end{minipage}
 \begin{minipage}{0.49\hsize}
        \centering
    \includegraphics[width=1\textwidth]{neurips2019/fig/D=3_L=100_N=1000.pdf}
    \subcaption{$D = 3$}    \label{fig:mse_vsM_d3}
 \end{minipage}
 \caption{Dimension of a simplex $M$ vs. MSE with $(L, N) = (100, 1000)$ (boxplots over 20 trials and theoretical risks).}
 \label{fig:mse_vsM}
\end{figure}

\Cref{fig:mse_vsL} shows box plots of MSEs over 20 trials and our theoretical risks for each $L\in \Set{8, 25, 50, 100}$ and $D = 2, 3$ with $(M, N)=(8, 1000)$.
As well as \cref{fig:mse_vsN,fig:mse_vsM}, the inductive skeleton fitting always outperform the all-at-once fitting in the case of $D = 2$.


\begin{figure}[ht]
 \begin{minipage}{0.49\hsize}
        \centering
    \includegraphics[width=1\textwidth]{neurips2019/fig/D=2_M=8_N=1000.pdf}
    \subcaption{$D = 2$}   
    \label{fig:mse_vsL_d2}
 \end{minipage}
 \begin{minipage}{0.49\hsize}
        \centering
    \includegraphics[width=1\textwidth]{neurips2019/fig/D=3_M=8_N=1000.pdf}
    \subcaption{$D = 3$}   
    \label{fig:mse_vsL_d3}
 \end{minipage}
 \caption{Dimension of control points $L$ vs. MSE with sample size $(M, N) = (3, 8, 1000)$ (boxplots over 20 trials and theoretical risks).}
\label{fig:mse_vsL}
\end{figure}

\subsection{Multi-objective optimization instances}\label{sec:MOP-instances}
Next, we describe the experiment protocol for the multi-objective optimization instances used in \cref{main:sec:MOP-instances} of the main paper.
We only present the data generation process since the fitting and evaluation part are the same as the synthetic cases.

\subsubsection{A generalized location problem}\label{sec:location-problem}
We generalized the multi-objective location problem \cite{Kuhn1967} to a higher dimension:
\begin{equation}
\begin{split}
\text{minimize } & f(x) = (f_1(x), f_2(x), f_3(x)) \text{ subject to }x \in \R^4\\
\text{where }    & f_m(x) = \norm{x - e_m}^2 \quad (m = 1, \dots, 3)\\
                 & e_1 = (1,0,0,0),\ e_2 = (0,1,0,0),\ e_3 = (0,0,1,0).
\end{split}
\end{equation}
Note that this is a special case of the MED benchmark problem \cite{Hamada2010}.
The MED problem is simplicial \cite{Hamada2017} and its Pareto set is known to be the convex hull of the minimizers of separate objective functions, i.e., the 2-simplex spanned by $e_1, e_2, e_3$.
For each vertex, edge, face of this simplex, which is the Pareto set of each 1-, 2-, 3-objective subproblem, we generate a subsample according to the uniform distribution on it.

\subsubsection{The group lasso}\label{sec:group-lasso}
We applied the B\'ezier simplex fittings to estimation of the hyper-parameter space of a sparse regression method.
The dataset used in this experiment was \texttt{Birthwt} in the R-package \texttt{MASS}, which contains 189 births at the Baystate Medical Centre, Springfield, Massachusetts during 1986 \cite{Hosmer1989,Venables2002}.
From the dataset, we adopted six continuous features \texttt{age1}, \texttt{age2}, \texttt{age3}, \texttt{lwt1}, \texttt{lwt2}, \texttt{lwt3} as predictors and one continuous feature \texttt{bwt} as a response for regression analysis.
Since the predictors are classified into two groups, \texttt{age} and \texttt{lwt}, the group lasso~\cite{Yuan2006} was employed.

Put $N=189$ and $M=6$.
Let $A$ be an $N \times M$ matrix of observations of the predictors, $x \in \R^M$ be a row vector of the predictor coefficients to be estimated, separated into two groups $x_\text{age} = (x_1, x_2, x_3)^\top$ and $x_\text{lwt} = (x_4, x_5, x_6)^\top$, and $y \in \R^N$ be a row vector of observations of the response.
The group lasso regressor is the solution to the following problem:
\begin{equation}\label{eqn:group-lasso}
\text{minimize } \frac{1}{2N} \norm{Ax - y}^2 + \frac{\lambda}{\sqrt{3}} \paren{\norm{x_\text{age}} + \norm{x_\text{lwt}}} \text{ subject to } x \in \R^6
\end{equation}
where $\norm{\cdot}$ is the Euclidean norm, and $\lambda$ is a positive number to be tuned by users.
This original form suffers from two drawbacks:
\begin{itemize}
    \item Choosing an appropriate value for $\lambda$ involves a grid search on an unbounded domain.
    \item Since two groups have physically different units of measurement, same weights are not always appropriate even if their values are normalized.
\end{itemize}

Instead, we consider each term in \cref{eqn:group-lasso} as a separate objective function:
\begin{equation}\label{eqn:group-lasso-mop}
\begin{split}
    \text{minimize } & f(x)   = (f_1(x), f_2(x), f_3(x)) \text{ subject to } x \in \R^6\\
    \text{where }    & f_1(x) = \norm{Ax - y}^2,\ f_2(x) = \norm{x_\text{age}}^2,\ f_3(x) = \norm{x_\text{lwt}}^2.
\end{split}
\end{equation}
Notice that the use of the squared norm in $f_2$ and $f_3$ does not change their solutions.
It is easy to see that every objective function in \cref{eqn:group-lasso-mop} is convex but not strongly convex.
We make them strongly convex by the following perturbation:
\begin{align*}
    \tilde f_1 &= f_1 + \varepsilon \norm{x}^2,\\
    \tilde f_2 &= f_2 + \varepsilon \norm{x}^2,\\
    \tilde f_3 &= f_3 + \varepsilon \norm{x}^2
\end{align*}
where $\varepsilon$ is an arbitrarily small positive number (we set $\varepsilon=10^{-4}$).
Now the problem minimizing a mapping $\tilde f= (\tilde f_1, \tilde f_2, \tilde f_3)$ is strongly convex.
By \cite[Theorems 1.1 and 3.1]{Hamada2019}, this problem is weakly simplicial and the mapping
\begin{equation}\label{eqn:group-lasso-sop}
    x^*(w) = \arg\min_x \inprod{w}{f(x)}
\end{equation}
is well-defined and continuous on $\Delta^2$, satisfying $x^*(\Delta^2_I) = X^*(\tilde f_I)$ for all $I \subseteq \set{1,2,3}$.

Then, we obtained subsamples by solving \cref{eqn:group-lasso-sop} repeatedly with varying $w \in \Delta^2_I$ for each $I \subseteq \set{1, 2, 3}$.
For each such $I$, the weight $w$ was drawn from the uniform distribution on $\Delta^2_I$ and the problem \cref{eqn:group-lasso-sop} was solved by the steepest descent method.

The same idea can be applied to a broad range of sparse learning methods, including the original lasso \cite{Tibshirani1996}, the fused lasso \cite{Tibshirani2005}, the smooth lasso \cite{Hebiri2011}, and the elastic net \cite{Zou2005}.
For those methods, their group-wise regularization terms can be considered as separate objectives, and the resulting problems would be many-objective (four-objective or more) where the all-at-once fitting will much outperform over the inductive skeleton fitting.
We however remark that the bridge regression \cite{Frank1993} is not the case since its regularization term using a nonconvex $\ell_p$-norm (i.e., $p < 1$) cannot change into a strongly convex function via perturbations.

\subsubsection{The Pareto fronts of each multi-objective optimization problem}\label{sec:Pareto-fronts}
\Cref{fig:Pareto-fronts} shows the Pareto fronts of the location problem and the group lasso.
From \cref{fig:Pareto-fronts}, we can see that the Pareto front of the location problem can be represented by a B\'ezier simplex of degree $D=2$.
For the group lasso on the other hand, its Pareto front cannot be represented by a a B\'ezier simplex of degree $D=2$ but of $D=3$. 
\begin{figure}[ht]
 \begin{minipage}{0.49\hsize}
        \centering
    \includegraphics[width=1\textwidth]{neurips2019/fig/pareto_med.pdf}
    \subcaption{Location problem}   
    \label{fig:Pareto-front-location-problem}
 \end{minipage}
 \begin{minipage}{0.49\hsize}
        \centering
    \includegraphics[width=1\textwidth]{neurips2019/fig/pareto_grouplasso.pdf}
    \subcaption{Group lasso}   
    \label{fig:Pareto-front-group-lasso}
 \end{minipage}
 \caption{The Pareto fronts of the location problem and the group lasso.}
\label{fig:Pareto-fronts}
\end{figure}

% \section{Application to Hyper-parameter Tuning of Elastic Net}\label{sec:application}
% The elastic net~\cite{Zou2005} is a regression method for linear models with multicollinearity and sparsity.
% Given a matrix $A$ with $m$ rows of observations and $n$ columns of features (may be $m \ll n$), a row vector $y$ of $m$ responses, the elastic net regression is the solution to the following problem:
% \begin{equation}\label{eqn:elastic-net}
% \text{minimize } \frac{1}{2N} \norm{Ax - y}^2 + \alpha \paren{\rho \abs{x} + \frac{1 - \rho}{2} \norm{x}^2} \text{ subject to } x \in \R^n
% \end{equation}
% where $\norm{\cdot}$ is the Euclidean norm, $\abs{\cdot}$ is the $\ell_1$-norm, and $\alpha, \rho \geq 0$ are fixed coefficients for regularization.
% Note that with $\alpha = 0$, it is the ordinary least squares regression; with $\alpha > 0$ and $\rho = 0$, it becomes the ridge regression \cite{Hoerl1970} which stabilizes a solution against multicollinear variables; with $\alpha > 0$ and $\rho = 1$, it turns into the lasso regression \cite{Tibshirani1996} which finds a sparse solution from a lot of features; when $\alpha > 0$ and $0 < \rho < 1$, it thus inherits both of their properties.
% Choosing appropriate values for $\alpha$ and $\rho$ involves a 2-D grid search on an unbounded domain, which often requires heavy computational effort.

% We consider each term in \cref{eqn:elastic-net} as an objective function:
% \begin{equation}\label{eqn:elastic-net-mop}
% \begin{split}
%     \text{minimize } & f(x)   = (f_1(x), f_2(x), f_3(x)) \text{ subject to } x \in \R^n\\
%     \text{where }    & f_1(x) = \norm{Ax - y}^2,\ f_2(x) = \abs{x},\ f_3(x) = \norm{x}^2.
% \end{split}
% \end{equation}
% While $f_3$ is strongly convex, $f_1$ and $f_2$ are (not strongly) convex.
% We make them strongly convex by the following perturbation:
% \begin{align}
% \tilde f_1 &= f_1 + \varepsilon_1 f_3,\\
% \tilde f_2 &= f_2 + \varepsilon_2 f_3
% \end{align}
% where $\varepsilon_1, \varepsilon_2 > 0$ are arbitrarily small numbers.
% Now the problem minimizing a mapping $\tilde f= (\tilde f_1, \tilde f_2, f_3)$ is strongly convex.
% This problem is weakly simplicial and the mapping
% \[
%  x^*(w) = \arg\min_x \inprod{w}{f(x)}
% \]
% is well-defined and continuous on $\Delta^2$ with $x^*(\Delta^2_I) = X^*(\tilde f_I)$ for all $I \subseteq \Set{1,2,3}$.

% We can obtain subsamples by solving each subproblem and can fit a B\'ezier triangle of degree two.
% By setting the weight as
% \[
% w = (1,0,0),(0,1,0),(0,0,1),(\frac 1 2, \frac 1 2, 0),(\frac 1 2, 0, \frac 1 2), (0, \frac 1 2, \frac 1 2),
% \]
% we obtain seven models and their objective values evaluated with test data.
% We obtain an approximate trade-off triangle between error, sparsity, and multicollinearity.

% We convert weights $(w_1, w_2, w_3) \in \Delta^2$ to regularization coefficients $(\alpha, \rho) \in \R_{\geq 0}^2$ for the \texttt{scikit-learn}'s elastic net as follows:
% \[
% \alpha = \frac{w_2 + w_3 + 2 \varepsilon}{w_1 + \varepsilon},\quad
% \rho = \frac{w_2 + \varepsilon}{w_2 + w_3 + 2 \varepsilon}
% \]
% where $\varepsilon > 0$ is a small number to avoid numerical instability (we set $\varepsilon=10^{-4}$).

% The same idea can be applied to a broad range of sparse learning methods, including the group LASSO \cite{Yuan2006}, the fused LASSO \cite{Tibshirani2005}, the smooth LASSO \cite{Hebiri2011}, and their elastic net counterparts.
% For those methods, their group-wise regularization terms can be considered as separate objectives, and the resulting problems would be many-objective (four-objective or more) where the inductive skeleton will much outperform over the all-at-once fitting.
% We however remark that the bridge regression \cite{Frank1993} is not the case since its regularization term using a nonconvex $\ell_p$-norm (i.e., $p < 1$) cannot change into a strongly convex function via perturbations.


\bibliographystyle{plain}
\bibliography{reference}
\end{document}
