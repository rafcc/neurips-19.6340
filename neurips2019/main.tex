\documentclass{article}
% if you need to pass options to natbib, use, e.g.:
%     \PassOptionsToPackage{numbers, compress}{natbib}
% before loading neurips_2019

% ready for submission
 \usepackage[preprint]{neurips_2019}

% to compile a preprint version, e.g., for submission to arXiv, add add the
% [preprint] option:
%     \usepackage[preprint]{neurips_2019}

% to compile a camera-ready version, add the [final] option, e.g.:
%     \usepackage[final]{neurips_2019}

% to avoid loading the natbib package, add option nonatbib:
%     \usepackage[nonatbib]{neurips_2019}

%%%%%%%%%%%%
%\usepackage[color]{showkeys}%ラベルを表示
\usepackage[utf8]{inputenc} % allow utf-8 input
\usepackage[T1]{fontenc}    % use 8-bit T1 fonts
\usepackage{url}            % simple URL typesetting
\usepackage{booktabs}       % professional-quality tables
\usepackage{amsfonts}       % blackboard math symbols
\usepackage{nicefrac}       % compact symbols for 1\2, etc.
\usepackage{microtype}      % microtypography
\usepackage{graphicx}
\usepackage{ascmac}
\usepackage{amsmath,amssymb,amsfonts,array}
\usepackage{theorem}
\usepackage{bm}
\usepackage{braket}
\usepackage{extarrows}
\usepackage{booktabs}
\usepackage{algorithm,algorithmic}
\usepackage{tikz}
\usepackage{color}
\usepackage{xcolor}
\usepackage{subcaption}
\usepackage{multirow}
\usepackage{xr-hyper}
\usepackage{hyperref}
\usepackage[capitalise,noabbrev]{cleveref}

%%%% for xy-hyper
\makeatletter
\newcommand*{\addFileDependency}[1]{% argument=file name and extension
  \typeout{(#1)}
  \@addtofilelist{#1}
  \IfFileExists{#1}{}{\typeout{No file #1.}}
}
\makeatother
 
\newcommand*{\myexternaldocument}[2][]{%
    \externaldocument[#1]{#2}%
    \addFileDependency{#2.tex}%
    \addFileDependency{#2.aux}%
}

%%%% for showkey of cleveref
% \makeatletter
%   \SK@def\Cref#1{\SK@\SK@@ref{#1}\SK@Cref{#1}}%
%   \SK@def\cref#1{\SK@\SK@@ref{#1}\SK@cref{#1}}%
% \makeatother

% \definecolor{refkey}{rgb}{0.9451,0.2706,0.4941}
% \definecolor{labelkey}{rgb}{0.9451,0.2706,0.4941}

%%%% for tikz
\usetikzlibrary{positioning}

%%%% for cleveref
\crefname{equation}{}{}

\theoremstyle{plain}
%\theoremstyle{definition}
\newtheorem{theorem}{Theorem}
\newtheorem{proposition}{Proposition}
\newtheorem{lemma}{Lemma}
\newtheorem{definition}{Definition}
\newtheorem{remark}{Remark}
\newtheorem{example}{Example}
\newtheorem{corollary}[theorem]{Corollary}
\newtheorem{proof}{Proof}
\renewcommand{\theproof}{}

\newtheorem{thm}{Theorem}[section]
\newtheorem{claim}{Claim}
\newtheorem{lem}{Lemma}[section]
\newtheorem{prop}{Proposition}[section]
\newtheorem{cor}{Corollary}[section]
\newtheorem{defn}{Definition}[section]
\newtheorem{exmp}{Example}[section]
%\renewcommand*{\proofname}{Proof}

%%%% commands
\newcommand{\norm}[1]{\left\| #1 \right\|}
\newcommand{\abs}[1]{\left| #1 \right|}
\newcommand{\card}[1]{\left| #1 \right|}
\newcommand{\paren}[1]{\left( #1 \right)}
\newcommand{\sqbra}[1]{\left[#1 \right]}
\newcommand{\inprod}[2]{\left\langle #1,\ #2 \right\rangle}
\newcommand{\R}{\mathbb{R}}
\newcommand{\N}{\mathbb{N}}
\newcommand{\E}{\mathbb{E}}
\newcommand{\diff}{\mathrm{d}}
\newcommand{\ZZ}{\Sigma}

\DeclareMathOperator{\rank}{rank}

\newcommand{\argmax}{\mathop{\rm arg~max}\limits}
\newcommand{\argmin}{\mathop{\rm arg~min}\limits}


\frenchspacing  %Required

% The \author macro works with any number of authors. There are two commands
% used to separate the names and addresses of multiple authors: \And and \AND.
%
% Using \And between authors leaves it to LaTeX to determine where to break the
% lines. Using \AND forces a line break at that point. So, if LaTeX puts 3 of 4
% authors names on the first line, and the last on the second line, try using
% \AND instead of \And before the third author name.

\author{%
  Akinori ~Tanaka%\thanks{Use footnote for providing further information
    %about author (webpage, alternative address)---\emph{not} for acknowledging
    %funding agencies.}
    \\
  RIKEN AIP, Keio University\\
  \texttt{akinori.tanaka@riken.jp} \\
  % examples of more authors
   \And
   Akiyoshi Sannai \\
   RIKEN AIP, Keio University\\
   \texttt{akiyoshi.sannai@riken.jp} \\
   \AND
   Ken Kobayashi \\
   Fujitsu Laboratories LTD., RIKEN AIP, Tokyo Tech\\
   \texttt{ken-kobayashi@fujitsu.com} \\
   \And
   Naoki Hamada \\
   Fujitsu Laboratories LTD., RIKEN AIP\\
   \texttt{hamada-naoki@fujitsu.com} \\
   %\And
   %Coauthor \\
   %Affiliation \\
   %Address \\
   %\texttt{email} \\
}

\myexternaldocument[apnd:]{appendix}
\title{Asymptotic Risk of B\'ezier Simplex Fitting}
\begin{document}
\maketitle

%%%%%%%%%%%%%%%%%%%%%%%%%%%%%%%%%%%%%%%%%%%%%%%%%%%%%%%%%%%%%%%%%%%%%%%%%%%%%%%%
\begin{abstract}
The B\'ezier simplex fitting is a novel data modeling technique which exploits geometric structures of data to approximate the Pareto front of multi-objective optimization problems.
There are two fitting methods based on different sampling strategies.
The \emph{inductive skeleton fitting} employs a stratified subsampling from each skeleton of a simplex, whereas the \emph{all-at-once fitting} uses a non-stratified sampling which treats a simplex as a whole.
In this paper, we analyze the asymptotic risks of those B\'ezier simplex fitting methods and derive the optimal subsample ratio for the inductive skeleton fitting.
It is shown that the inductive skeleton fitting with the optimal ratio has a smaller risk when the degree of a B\'ezier simplex is less than three.
Those results are verified numerically under small to moderate sample sizes.
In addition, we provide two complementary applications of our theory: a generalized location problem and a multi-objective hyper-parameter tuning of the group lasso.
The former can be represented by a B\'ezier simplex of degree two where the inductive skeleton fitting outperforms.
The latter can be represented by a B\'ezier simplex of degree three where the all-at-once fitting gets an advantage.
\end{abstract}

%%%%%%%%%%%%%%%%%%%%%%%%%%%%%%%%%%%%%%%%%%%%%%%%%%%%%%%%%%%%%%%%%%%%%%%%%%%%%%%%
\section{Introduction}
Given functions $f_1,\dots,f_M: X \to \R$ on a subset $X$ of the Euclidean space $\R^N$, consider the multi-objective optimization problem
\[
\text{minimize } f(x) := (f_1(x), \dots, f_M(x)) \text{ subject to } x \in X (\subseteq \R^N)
\]
with respect to the Pareto ordering: $x \prec y \xLeftrightarrow{\mathrm{def}} \forall i \sqbra{f_i(x) \leq f_i(y)} \land \exists j \sqbra{f_j(x) < f_j(y)}$.
The goal is to find the \emph{Pareto set} and its image, called the \emph{Pareto front}, which are denoted by
\[
X^*(f) := \Set{x \in X | \forall y \in X \sqbra{y \not\prec x}}
\quad \text{and} \quad
f(X^*(f)) := \Set{f(x) \in \R^M | x \in X^*(f)},
\]
respectively.
Since most numerical optimization approaches give a finite number of points as an approximation of those objects (e.g., goal programming \cite{Miettinen1999,Eichfelder2008}, evolutionary computation \cite{Deb2001,Zhang2007,Deb2014}, homotopy methods \cite{Hillermeier2001,Harada2007}, Bayesian optimization \cite{Hernandez-Lobato2016,Yang2019}), the complete shapes of them are usually not revealed.
To amplify the knowledge extracted from their point approximations, we consider in this paper a fitting problem of the Pareto set and front.

It is known that those objects often have skeleton structures that can be used to enhance fitting accuracy.
An $M$-objective problem is \emph{simplicial} if the Pareto set and front are homeomorphic to an $(M-1)$-dimensional simplex and each $(m-1)$-dimensional subsimplex corresponds to the Pareto set of an $m$-objective subproblem for all $0 \le m \le M$ (see \cite{Hamada2019} for precise definition and examples).
There are a lot of practical problems being simplicial: location problems \cite{Kuhn1967} and a phenotypic divergence model in evolutionary biology \cite{Shoval2012} are shown to be simplicial, and an airplane design \cite{Mastroddi2013} and a hydrologic modeling \cite{Vrugt2003} hold numerical solutions which imply those problems are simplicial.
The Pareto set and front of any simplicial problem can be approximated with arbitrary accuracy by a B\'ezier simplex of an appropriate degree~\cite{Kobayashi2019}.
There are two fitting algorithms for B\'ezier simplices: the all-at-once fitting is a na\"ive extension of Borges-Pastva algorithm for B\'ezier curves~\cite{Borges2002}, and the inductive skeleton fitting~\cite{Kobayashi2019} exploits the skeleton structure of simplicial problems discussed above.

An important problem class which is (generically) simplicial is the strongly convex problem.
It has been shown that many practical problems can be considered as strongly convex via appropriate transformations preserving the essential problem structure, i.e., the Pareto ordering and the topology~\cite{Hamada2019}.
For example, the multi-objective location problem \cite{Kuhn1967} can be strongly convex by squaring each objective function.
The resulting problem has a Pareto front that can be represented by a B\'ezier simplex of degree two \cite{Hamada2019}.
As we will show in this paper, the group lasso \cite{Yuan2006} can be reformulated as a simplicial problem.
It has a cubic Pareto front that requires a B\'ezier simplex of degree three.
The same transformation can be applied to a broad range of sparse learning methods, including the (original) lasso \cite{Tibshirani1996}, the fused lasso \cite{Tibshirani2005}, the smooth lasso \cite{Hebiri2011}, and the elastic net \cite{Zou2005}.
Since the required degree is problem-dependent, we need to understand the performance of the two B\'ezier simplex fittings with respect to the degree.

In this paper, we study the asymptotic risk of the two fitting methods of the B\'ezier simplex: the all-at-once fitting and the inductive skeleton fitting, and compare their performance with respect to the degree.

Our contributions are as follows:
\begin{itemize}
\item We have evaluated the asymptotic $\ell_2$-risk, as the sample size tends to infinity, of two B\'ezier simplex fitting methods: the all-at-once fitting and the inductive skeleton fitting.
\item In terms of minimizing the asymptotic risk, we have derived the optimal ratio of subsample sizes for the inductive skeleton fitting.
\item We have shown when the inductive skeleton fitting with optimal ratio outperforms the all-at-once fitting when the degree of a B\'ezier simplex is two, whereas the all-at-once has an advantage at degree three.
\item We have demonstrated that the location problem and the group lasso are transformed into strongly convex problems, and their Pareto fronts are approximated by a B\'ezier simplex, which numerically verifies the asymptotic results.
\end{itemize}

The rest of this paper is organized as follows:
\cref{sec:problem-definition} describes the problem definition.
\Cref{sec:asymptotic-risk} analyzes the asymptotic risks of the all-at-once fitting and the all-at-once fitting.
For the inductive skeleton fitting, the optimal subsample ratio in terms of minimizing the risk is derived.
Those analyses are verified in \cref{sec:numerical-examples} via numerical experiments.
\Cref{sec:conclusion} concludes the paper and addresses future work.

%%%%%%%%%%%%%%%%%%%%%%%%%%%%%%%%%%%%%%%%%%%%%%%%%%%%%%%%%%%%%%%%%%%%%%%%%%%%%%%%
\section{Problem definition}\label{sec:problem-definition}
Let $M$ be a non-negative integer.
The \emph{standard $(M - 1)$-simplex} is defined by
\[
\Delta^{M - 1} = \Set{(w_1, \dots, w_M) \in \R^M | \sum_{m = 1}^M w_m = 1,\ w_m \geq 0}.
\]
For an index set $I \subseteq \set{1, \dots, M}$, we define the \emph{$I$-subsimplex} of $\Delta^{M - 1}$ by $\Delta^I = \set{(w_1, \dots, w_M) \in \Delta^{M - 1} | w_m = 0\ (m \not \in I)}$.
For an integer $0 \leq m \leq M$, the \emph{$(m - 1)$-skeleton} of $\Delta^{M - 1}$ is defined by
\[
\Delta^{(m-1)} = \bigcup_{I \subseteq \set{1, \dots, M} \text{ s.t. } \card{I}=m} \Delta^{M-1}_I.
\]

\subsection{B\'ezier simplex and its fitting methods}\label{sec:bezier-simplex}
Let $\N$ be the set of non-negative integers (including zero!) and $M, D \in \N$.
We denote a simplex lattice by $\N_D^M := \set{(d_1,\dots,d_M) \in \N^M | \sum_{m=1}^M d_m = D}$.
Given the \emph{control points} $\bm p_{\bm d} \in \R^L$ $(\bm d \in \N_D^M)$, an \emph{$(M - 1)$-B\'ezier simplex of degree $D$} is a mapping $\bm b(\bm t): \Delta^{M-1}\to\R^L$ defined by
\begin{equation}\label{eqn:bezier-simplex}
\bm b(\bm t) := \sum_{\bm d\in\N_D^M} \binom{D}{\bm d} \bm t^{\bm d} \bm p_{\bm d}
\end{equation}
where $\binom{D}{\bm d} := \frac{D!}{d_1! d_2! \cdots d_M!}$, and for each $\bm t := (t_1, \dots, t_M) \in \R^M$ and $\bm d := (d_1, \dots, d_M) \in \N^M$, $\bm t^{\bm d}$ is a monomial $t^{d_1}_1 t^{d_2}_2 \cdots t^{d_M}_M$.

Kobayashi \text{et al.} \cite{Kobayashi2019} proposed two B\'ezier simplex fitting algorithms: the all-at-once fitting and the inductive skeleton fitting.
They are different in not only fitting algorithm but also sampling strategy.
The all-at-once fitting requires a training set $S_N:=\set{(\bm t_n, \bm x_n) \in \Delta^{M-1} \times \R^L | n = 1, \dots, N}$ and adjusts all control points at once by minimizing the ordinary least square loss: $\frac{1}{N}\sum_{n=1}^N \norm{\bm x_n - \bm b(\bm t_n^{(m)})}^2$.

The inductive skeleton fitting, on the other hand, requires skeleton-wise sampled training sets $S_{N^{(m)}} := \set{(\bm t^{(m)}_n, \bm x^{(m)}_n) \in \Delta^{(m)} \times \R^L | n = 1, \dots, N^{(m)}}~(m=0,\ldots,M-1)$.
It also divides control points as $\bm p_{\bm d}^{(m)}$ such that $\bm d$ has $m+1$ non-zero elements.
Such $\bm p_{\bm d}^{(m)}$ determine $m$-skeleton of a B\'ezier simplex.
The inductive skeleton fitting inductively adjusts $\bm p_{\bm d}^{(m)}$ from $m = 0$ to $M - 1$ by minimizing the ordinary least square loss of the $m$-skeleton $\frac{1}{N^{(m)}}\sum_{n=1}^{N^{(m)}} \norm{\bm x^{(m)}_n - \bm b(\bm t_n)}^2$.

\subsection{The \texorpdfstring{$\ell_2$}{l2}-risk}
The fitting problem considered in this paper is as follows.
The sample is taken from an unknown B\'ezier simplex $\bm b(\bm t): \Delta^{M-1} \to \R^L$ with additive Gaussian noise $\bm \varepsilon \sim N(\bm 0, \sigma^2 \bm I)$, that is, $\bm x = \bm b(\bm t) + \bm \varepsilon$.
For the all-at-once fitting, $S_N = \set{(\bm t_n, \bm x_n)}$ follows the uniform distribution on the domain of the B\'ezier simplex: $\bm t_n \sim U(\Delta^{M-1})$ and $\bm x_n = \bm b(\bm t_n) + \bm \varepsilon_n$.
For the inductive skeleton fitting, $S_{N^{(m)}} = \set{(\bm t_n^{(m)}, \bm x_n^{(m)})}$ follows the uniform distribution on the $m$-skeleton of the domain of the B\'ezier simplex: $\bm t_n^{(m)} \sim U(\Delta^{(m)})$ and $\bm x_n^{(m)} = \bm b(\bm t_n^{(m)}) + \bm \varepsilon_n^{(m)}$.
A B\'ezier simplex estimated from $S_N$ is denoted by $\bm{\hat b}(\bm t | S_N)$.
For both method, we asymptotically evaluate the $\ell_2$-risk below as $N \to \infty$.
\begin{equation}\label{eqn:risk-def}
R_N := \E_{S_N}\sqbra{\E_{\bm t \sim U(\Delta^{M-1})} \norm{\bm b(\bm t) - \hat{\bm b}(\bm t | S_N)}^2}.
\end{equation}
Here, we put $S_N = S_{N^{(0)}} \cup \dots \cup S_{N^{(M-1)}}$ for the inductive skeleton fitting.

%%%%%%%%%%%%%%%%%%%%%%%%%%%%%%%%%%%%%%%%%%%%%%%%%%%%%%%%%%%%%%%%%%%%%%%%%%%%%%%%
\section{Asymptotic risk of B\'ezier simplex fitting}\label{sec:asymptotic-risk}
To calculate the risk of each fitting scheme, let us first focus on the fact: the summation/subtraction of two B\'ezier simplices is also B\'ezier simplex.
In the definition \cref{eqn:risk}, we have $\bm b(\bm t) - \hat{\bm b}(\bm t | S_N)$ which measures the difference between the target B\'ezier simplex and the model B\'ezier simplex.
By using the above fact, $\bm b(\bm t) - \hat{\bm b}(\bm t | S_N)$ is also B\'ezier simplex.
Let us call its control point as $\bm p'$, and consider the following matrix,
\begin{align}
\bm P
=
\begin{bmatrix}
(\bm p_1')_1 &
(\bm p_1')_2 & 
\cdots &
(\bm p_1')_L 
\\
(\bm p_2')_1 &
(\bm p_2')_2 & 
\cdots &
(\bm p_2')_L 
\\
\quad \vdots&
\quad \vdots & 
\ddots &
\quad \vdots
\\
(\bm p_{\card{\N_D^M}}')_1 &
(\bm p_{\card{\N_D^M}}')_2 & 
\cdots &
(\bm p_{\card{\N_D^M}}')_L 
\end{bmatrix}
,
\end{align}
where $(\bm p_A)_{l}$ means the $l$-th component of the $A$-th control point vector $\bm p_A$.
%%%
The asymptotic rick can be calculated by the following theorem.
\begin{theorem}\label{thm:risk}
The risk of the B\'ezier simplex fitting can be represented by
\begin{align}
R_N
=
\sum_{\bm d_A, \bm d_B \in \mathbb{N}_D^M}
\ZZ_{AB}
\E_{S_N} \sqbra{({\bm P} {\bm P}^\top)_{AB}}
,
\label{eqn:risk}
\end{align}
where the matrix $\ZZ$ is defined by
\begin{align}
\ZZ_{AB}
=
\frac{(2D)!(M-1)!}{(2D+M-1)!}
  \binom{D}{\bm d_A}
  \binom{D}{\bm d_B}
  \binom{2D}{\bm d_A + \bm d_B}^{-1}
  \label{eqn:ZZ-def}
\end{align}
\end{theorem}
The proof is provided in the supplementary materials (\cref{apnd:sec:proof-of-main-theorem}).
Once the set of parameters of the system, including the simplex dimension $M$, degree of the B\'esier simplex $D$, the dimension of the target data $L$, the amplitude of the noise $\sigma$, is fixed, the equation \cref{eqn:risk} says that the asymptotic value of this risk function depends only on how we choose the matrix $\bm P$.
We calculate the asymptotic form of the risk \cref{eqn:risk} with $\bm P$ determined from the all-at-once (AAO) fitting and the inductive-skeleton (ISK) fitting.
We call them $\bm P_\mathrm{AAO}$ and $\bm P_\mathrm{ISK}$ respectively.

\subsection{All-at-once fitting}\label{sec:all-at-once-fitting}
\paragraph{Samples and determined control points}
Let us recall the sample $S_N$ consists of elements $(\bm t_n, \bm x_n) \in \Delta^{M-1} \times \R^L$ with $\bm x_n = \bm b(\bm t_n) + \bm \varepsilon_n$ and $\bm t_n \sim U(\Delta^{M-1})$, $\bm \varepsilon_n \sim N(0, \sigma^2 \bm I)$ for $n = 1, \dots, N$.
The matrix $\bm P_\mathrm{AAO}$ is determined by minimization of the OLS error below
\begin{align}
\frac 1 N \sum_{n=1}^N \norm{\bm x_n - \hat{\bm b}(\bm t_n)}^2 
%&= \frac 1 N \sum_{n=1}^N \norm{\sum_{\bm d\in\N_D^M} \binom{D}{\bm d}\bm t_n^{\bm d}  \bm p'_{\bm d} + \bm \varepsilon_n}^2\notag\\
&= \frac 1 N \norm{ \bm Z \bm P + \bm Y }_{\mathrm F}^2, \label{eqn:rss}
\end{align}
where $\norm{\cdot}_\mathrm{F}$ means the Frobenius norm and
$\bm z_n = [\text{a vector with component } \binom{D}{\bm d}\bm t_n^{\bm d}~(\bm d\in \N_D^M)] \in \R^{\card{\N_D^M}}$, $\bm Z = \sqbra{\bm z_1 \bm z_2 \cdots \bm z_N}^\top \in \R^{N \times \card{\N_D^M}}$, $\bm Y = \sqbra{\bm \varepsilon_1 \bm \varepsilon_2\cdots \bm \varepsilon_N}^\top \in \R^{N \times L}$.
In this notation, the optimum takes well known form: $\bm P_\mathrm{AAO} = -\paren{\bm Z^\top \bm Z}^{-1}\bm Z^\top \bm Y$.
Note that the regularity of the matrix $\bm Z^\top \bm Z$ is guaranteed by taking a sufficiently large number of samples $S_N$, or more precisely $\set{\bm t_n}_{n = 1, \dots, N}$.

%%%
\paragraph{Calculation of the asymptotics}
To calculate the risk asymptotics \cref{eqn:risk} in the all-at-once fitting, we need to calculate asymptotic values of expectation values of the matrix $\bm P_\mathrm{AAO} \bm P_\mathrm{AAO}^\top$ over $S_N$.
%
The first observation is that the contribution from the noise matrix $\bm Y$ is only located at the middle of the sequence of matrix product $\bm P_\mathrm{AAO} \bm P_\mathrm{AAO}^\top = \paren{\bm Z^\top \bm Z}^{-1}\bm Z^\top \bm Y \bm Y^\top \bm Z \paren{\bm Z^\top \bm Z}^{-1}$.
If $\bm Y \bm Y^\top \propto 1_N$, the calculation reduces very simple form.
In fact, we can perform it by decomposing $\E_{S_N}$ to $\E_{\bm t_n} \E_{\bm \varepsilon_n}$.
After taking expectation value $\E_{\bm \varepsilon_n}$,
we get
\begin{align}
\E_{S_N} \sqbra{\bm P_\mathrm{AAO} \bm P_\mathrm{AAO}^\top}
&= %%%%
\sigma^2 L \cdot
\E_{\bm t_n} \sqbra{
\paren{\bm Z^\top \bm Z}^{-1}
}
.
\label{eqn:P_AAO-P_AAO}
\end{align}
The prefactor $\sigma^2 L$ results from taking expectation over the noise.
Here, $L$ is the dimension of the space control points lived in.
%
Now, the only remaining task is the estimation of the asymptotic behavior of the matrix $(\bm Z^\top \bm Z)$.
%
A key observation is that each element of this matrix is an average over the samples $\bm t_n$:
\begin{align}
\frac{1}{N}
\paren{\bm Z^\top \bm Z}_{AB}
=
\binom{D}{\bm d_A}
\binom{D}{\bm d_B}
\sum_{n=1}^N 
\frac{1}{N}
\bm t_n^{\bm d_A + \bm d_B}
.
\end{align}
In fact, it converges to the matrix $\ZZ_{AB}$ defined in \cref{eqn:ZZ-def} as $N \to \infty$.
Therefore, by using the law of large numbers, we can get 
$
\paren{\bm Z^\top \bm Z}_{AB}
\overset{p}{\to}
N \ZZ_{AB}.
$
To substitute it to \cref{eqn:P_AAO-P_AAO}, we need to guarantee $\ZZ_{AB}$ has the inverse matrix, i.e.
\begin{theorem}\label{thm:sigma-is-non-singular}
For any $D, M$, the matrix $\ZZ_{AB}$ in \cref{eqn:ZZ-def} is non-singular.
\end{theorem}
The proof is given in the supplementary materials (\cref{apnd:sec:proof-of-regularity}).
Just by replacing $(\bm Z^T \bm Z)$ to $N \ZZ$, we arrive at the asymptotic form of the risk.
In addition to it, we can further simplify the result by using: $\sum_{AB} \ZZ_{AB}
\ZZ^{-1}_{AB} = \ _{D+M-1} C_D$, which is relatively easy to show (see \cref{apnd:sec:derivation-of-sum} in the supplementary materials).
\if0
\footnote{
Let $\tilde{\ZZ}$ be a cofactor matrix and $C_{AB}$ be a $(A, B)$ minor of $\ZZ$.
\begin{equation*}
\begin{split}
\sum_{A, B} \ZZ_{AB} \ZZ_{AB}^{-1}
    &= \sum_{A,B} \left(\frac{1}{\det(\ZZ)}\tilde{\ZZ} \circ \ZZ\right)_{AB} %\\
    %&= \frac{1}{\det(A)}\sum_{i, j = 1}^N (\tilde{A}\circ A)_{ij} \\
    %&
    = \frac{1}{\det(\ZZ)}\sum_{A,B} \ZZ_{AB}C_{AB} \\
    &= \frac{1}{\det(\ZZ)}\sum_{A}^{} \det(\ZZ)
    =  \ _{D+M-1} C_D.
\end{split}
\end{equation*}
From 1st line to 2nd line, we use the definition of the determinant of $\ZZ$.
}
\fi

In summary, our formula for the asymptotic form of the risk for the all-at-once fitting is
\begin{align}
R_N \overset{p}{\to}
\frac{\sigma^2 L}{N}
\sum_{A, B}
\ZZ_{AB}
\ZZ^{-1}_{AB}
=
\frac{\sigma^2 L}{N}
\ _{D+M-1} C_D \quad \text{as $N \to \infty$}.
\label{eqn:risk-all-at-once}
\end{align}

%%
\subsection{Inductive skeleton fitting}\label{sec:inductive-skeleton-fitting}
Before showing the asymptotic form of the risk for the inductive skeleton fitting, it would be better to introduce some notations here.
To treat subsimplices of the simplex $\Delta^{M-1}$, it is useful to notice that there is a one-to-one correspondence between a subsimplex and $M$-dimensional binary vector:
\begin{align}
\text{a subsimplex of $\Delta^{M-1}$}
\quad
\Longleftrightarrow
\quad
\bm I = [I_1, I_2, \dots, I_M], \quad
I_i \in \Set{0, 1}, \quad
\bm I \neq \bm 0.
\end{align}
In this notation, $\Delta^{M-1}$ itself is identified to $[1,1, \dots, 1]$.
The sum $\card{\bm I} = \sum_{i=1,2, \dots, M} I_i$ provides the dimension + 1 of the corresponding subsimplex.
We call a subsimplex indexed by $\bm I$ as $\Delta^{\bm I}$ from now on.
In addition, it is useful to define notation for the set of all $(m-1)$-dimensional subsimplices:
\begin{align}
(m) = \text{all $(m)$-dimensional subsimplices of $\Delta^{M-1}$}
=
\cup_{\card{\bm I} = m+1} \Delta^{\bm I}
,
\end{align}
and we call corresponding control point submatrices as $\bm P^{(m)}$.
%

\paragraph{Samples and determined control points}
In this notation, we can state that the inductive skeleton fitting is an inductive procedure of determining control points matrices $\bm P^{(m)}$ from low $m=0, 1, \dots, M-1$.
%
Suppose all $\set{\bm P^{(k)}}_{k < m}$ are already fixed and the samples on $\cup_{\card{\bm I} = m} \Delta^{\bm I}$,
$S_{N^{(m)}} = \set{(\bm t_1^{(m)}, \bm x_1^{(m)}), \dots, (\bm t_{N^{(m)}}^{(m)}, \bm x_{N^{(m)}}^{(m)})}$ are provided from
${\bm t}_n^{(m)} \sim U(\cup_{\card{\bm I} = m+1} \Delta^{\bm I})$.
The $m$-th submatrix $\bm P^{(m)}$ is determined by minimizing the OLS error
\begin{align}
%&\frac{1}{N^{(m)}}
%\sum_{n = 1}^{N^{(m)}}
%\norm{ {\bm x}_{n}^{(m)} - 
%\hat{\bm b} ({\bm t}_{n}^{(m)}) }^2
%\notag \\
%&= %%%%
&
\frac{1}{N^{(m)}}
\sum_{n = 1}^{N^{(m)}}
\norm{
\bm x_n^{(m)}
-
\hat{\bm b}(\bm t_n^{(m)})
}^2\label{eqn:OLS}
\end{align}
Note that there is no need to take any control point on $\Delta^{\bm J}$ with $\card{\bm J} > m$ into account because each $\bm t_n^{(m)}$ is on $\Delta^{\bm I}$ with $\card{\bm I} = m+1$ and there is no contribution to $\hat{\bm b}(\bm t_n^{(m)})$ from such higher dimensional control point.
In addition, we regard lower dimensional control points already fixed, so the net objective control points are ones included in $\bm P^{(m)}$.
By repeating similar procedure done in the all-at-once fitting, we can conclude $\bm P^{(m)}$ is determined as
\begin{align}
{\bm P}_\mathrm{OLS}^{(m)}
= %%%%
-
[({\bm Z}^{(m)})^\top {\bm Z}^{(m)}]^{-1}
({\bm Z}^{(m)})^\top 
\paren{
{\bm Y}^{(m)}
+
\sum_{k < m}
{\bm Z}^{(m) [k]}
{\bm P}_\mathrm{OLS}^{(k)}
}
\label{eqn:P_OLS}
\end{align}
where
$
{\bm z}_{n}^{(m)[k]}
=
\sqbra{\text{a vector with component }
({\bm z}_{n}^{(m)})^{{\bm d}^{(k)}}
}$,
$
{\bm Z}^{(m) [k]}
=
\sqbra{{\bm z}_1^{(m) [k]}
 \bm{z}_2^{(m) [k]}
 \cdots 
 {\bm z}_{N^{(m)}}^{(m) [k]}}^\top$,
$
{\bm Y}^{(m)}
=
\sqbra{{\bm \varepsilon}_1^{(m)} \bm{\varepsilon}_2^{(m)} \cdots {\bm \varepsilon}_{N^{(m)}}^{(m)}}^\top
.
$
%%%

\paragraph{Calculation of the asymptotics}
We get ${\bm P_\mathrm{ISK}} {\bm P_\mathrm{ISK}}^\top
=
\oplus_{i, j=0}^{M-1}
{\bm P}^{(i)}_\mathrm{OLS}
({\bm P}^{(j)}_\mathrm{OLS})^\top
$ which we need to compute the risk \cref{eqn:risk}.
%This matrix is determined from lower sub matrices.
%
As one might notice, the risk for the inductive-skeleton fitting depends on each number of $(m)$-dimensional subsamples $N^{(m)}$.
We will determine the best combination of $N^{(m)}$ constrained on $\sum_m N^{(m)} = N$ later.
Here, we treat the risk depending not $N$ but every $N^{(m)}$ and call it as
$
R_{N^{(0)}, N^{(1)}, \dots, N^{(M-1)}}
$.
%
To calculate $\E_{S_N} [\bm P_{ISK} \bm P_{ISK}^\top]$, we again take expectation over noise.
Thanks to $\E [\bm Y^{(m)} (\bm Y^{(n)})^\top ] = \sigma^2 L \bm 1_{N^{(m)}}$ or $\bm 0$ depending on $m=n$ or not, and the central limit with respect to $\bm z_n$, one can get
\begin{align}
&\E_{ S_N }
\sqbra{
{\bm P}^{(i)}_\mathrm{OLS}
({\bm P}^{(j)}_\mathrm{OLS})^\top
}
\notag \\
&\overset{p}{\to}
\sigma^2 L
\sum_{
\substack{
m \leq i
\\
m \leq j
}}
\sum_{
\substack{
m \leq k_1 < \dots < k_{\heartsuit} < i
\\
m \leq l_1 < \dots < l_{\spadesuit} < j
}
}
\frac{(-1)^{\heartsuit + \spadesuit}}{N^{(m)}}
%\notag \\ \nline \quad \times %\qquad %\qquad
{\bm \Lambda}_{(i)}
{\bm \Lambda}^{(i)[k_{\heartsuit}]}
{\bm \Lambda}_{(k_{\heartsuit} )}
%\hat{\bm \Lambda}^{(k_{\#_i})[k_{\#_i -1}]}
\cdots
{\bm \Lambda}^{(k_1)[m]}
{\bm \Lambda}_{(m)}
{\bm \Lambda}^{[m](l_1)}
\cdots
{\bm \Lambda}_{(l_{\spadesuit} )}
{\bm \Lambda}^{[l_{\spadesuit}](j)}
{\bm \Lambda}_{(j)}
\label{eqn:P_OLS-P_OLS}
\end{align}
after substituting the recursive formula \cref{eqn:P_OLS} repeatedly, where
\begin{align}
&({\bm \Lambda}^{(m)[k]})_{ \bm d ^{(m)} \bm d ^{(k)} }
%\notag \\
= %%%%
\frac{(m-1)!}{\ _{M} C_m}
\begin{pmatrix}
D \\
{\bm d}^{(m)}
\end{pmatrix}
%
\begin{pmatrix}
D \\
{\bm d}^{(k)}
\end{pmatrix}
%
\sum_{ \card{I} = m }
\frac{
\delta_{{\bm I}, (\bm d ^{(m)} +  \bm d ^{(k)} )_{01} }
\prod_{I_i = 1}( \bm d ^{(m)} +  \bm d ^{(k)}  )_i !}{
[\sum_{I_i = 1} ( \bm d ^{(m)} +  \bm d ^{(k)}  )_i 
+ m - 1]!
},
\notag
\\
&
{\bm \Lambda}^{[k](m)}
=
{\bm \Lambda}^{(m)[k]},
\quad
{\bm \Lambda}_{(m)}
%\notag \\
= 
({\bm \Lambda}^{(m)[m]})^{-1}
\label{eqn:Lamb}
\end{align}
and
\begin{align}
\delta_{{\bm I}, (\bm d ^{(m)} +  \bm d ^{(k)} )_{01} }
&=
\left\{ \begin{array}{ll}
1 & ( {\bm I} = (\bm d ^{(m)} +  \bm d ^{(k)} )_{01}  ) \\
0 & \text{otherwise}\\
\end{array} \right.
\end{align}
For the complete derivation, see \cref{apnd:sec:risk-derivation} in the supplementary materials.
We stop here and leave to get a closed formula for the asymptotics of the risk on inductive-skeleton fitting as future work.
Instead, we calculate the asymptotic risk numerically by using \cref{eqn:P_OLS-P_OLS,eqn:risk} and obtain \cref{tab:risk-inductive}.
%%
\begin{table}[H]
\centering
\caption{Numerically computed asymptotic risks of the inductive skeleton fitting ($M$: dimension of B\'ezier simplex, $D$: degree of B\'ezier simplex, $N^{(m)}$: sample size of $(m)$-skeleton).}\label{tab:risk-inductive}
\begin{tabular}{l|ll} \toprule
$R_{N^{(0)}, N^{(1)}, \dots}$ &
$D = 2$ &
$D = 3$
\\ \midrule
$M = 2$ &
$1.0 / N^{(1)} + 0.5 / N^{(0)}$ &
$2.0 / N^{(1)} + 0.2666 / N^{(0)}$
\\
$M = 3$ &
$3.0 / N^{(1)} + 0.375 / N^{(0)}$ &
$1.0 / N^{(2)} + 3.535 / N^{(1)} + 0.1464 / N^{(0)}$
\\
$M = 4$ &
$5.142 / N^{(1)} + 0.4571 / N^{(0)}$ &
$5.333 / N^{(2)} + 4.714 / N^{(1)} + 0.1650 / N^{(0)}$
\\
$M = 5$ &
$7.142 / N^{(1)} + 0.625 / N^{(0)}$ &
$13.33 / N^{(2)} + 6.666 / N^{(1)} + 0.2083 / N^{(0)}$
\\
$M = 6$ &
$8.928 / N^{(1)} + 0.8214 / N^{(0)}$ &
$24.24 / N^{(2)} + 9.740 / N^{(1)} + 0.2575 / N^{(0)}$
\\
$M = 7$ &
$10.5 / N^{(1)} + 1.020 / N^{(0)}$ &
$37.12 / N^{(2)} + 13.84 / N^{(1)} + 0.3119 / N^{(0)}$
\\
$M = 8$ &
$11.87 / N^{(1)} + 1.212 / N^{(0)}$ &
$51.17 / N^{(2)} + 18.73 / N^{(1)} + 0.3723 / N^{(0)}$
\\ \bottomrule
\end{tabular}
\end{table}

%%
\subsection{All-at-once vs Inductive skeleton}\label{sec:all-vs-inductive}
\Cref{tab:risk-inductive} tells the risk of the inductive skeleton fitting depends on subsample sizes $N^{(m)}$.
Given total sample size $N$, we can minimize the risk by finding the optimally-decoupled subsample sizes:
\begin{align}
    R_N := \min_{N^{(0)}, \dots, N^{(M-1)}} \Set{R_{N^{(0)}, \dots, N^{(M-1)}}} \text{ subject to } \sum_{m=0}^{M-1} N^{(m)} = N.
\end{align}
We calculated optimal risks for all cases shown in \cref{tab:risk-inductive} and compared them to the risks of the all-at-once fitting.
\Cref{tab:risk-comparison} shows the results.
\begin{table}[H]
\centering
\caption{Comparison of asymptotic risks of the all-at-once $R_N^\mathrm{AAO}$ vs the inductive skeleton with the optimal subsample ratio $R_N^\mathrm{ISK}$ ($M$: dimension of B\'ezier simplex, $D$: degree of B\'ezier simplex, $N$: sample size). The winner is shown in bold.}\label{tab:risk-comparison}
\begin{tabular}{l|ll|ll} \toprule
& \multicolumn{2}{c|}{$D = 2$} & \multicolumn{2}{c}{$D = 3$} \\ 
& $R_{N}^\mathrm{AAO}$ & $R_N^\mathrm{ISK}$
& $R_{N}^\mathrm{AAO}$ & $R_N^\mathrm{ISK}$
\\ \midrule
$M = 2$ &
$3.0 / N$ &
$\bm{2.91421} / N$ &
$4.0 / N$ &
$\bm{3.72726} / N$
\\
$M = 3$ &
$6.0 / N$ &
$\bm{5.49632} / N$ &
$\bm{10.0} / N$ &
$10.6472 / N$
\\
$M = 4$ &
$10.0 / N$ &
$\bm{8.66660} / N$ &
$\bm{20.0} / N$ &
$23.8821 / N$
\\
$M = 5$ &
$15.0 / N$ &
$\bm{11.9936} / N$ &
$\bm{35.0} / N$ &
$44.7548 / N$
\\
$M = 6$ &
$21.0 / N$ &
$\bm{15.1663} / N$ &
$\bm{56.0} / N$ &
$73.1387 / N$
\\
$M = 7$ &
$28.0 / N$ &
$\bm{18.0687} / N$ &
$\bm{84.0} / N$ &
$107.570 / N$
\\
$M = 8$ &
$36.0 / N$ &
$\bm{20.6799} / N$ &
$\bm{120.0} / N$ &
$146.206 / N$
\\ \bottomrule
\end{tabular}
\end{table}

As one can see, the optimum inductive skeleton fitting outperforms the all-at-once fitting in $D = 2$, but it is not always correct in $D = 3$.
On $D = 2$, in fact, we can show that the minimum value of the inductive skeleton always less than the asymptotic risk of the corresponding all-at-one fitting.

% \begin{proposition}
% Let $N$ be a natural number and $a, b$ be positive real numbers. Let $f(x) = \frac{a}{x} + \frac{b}{N - x}$.
% Assume $a > b$ and $\frac{a - \sqrt{ab}}{a - b} < 1$.
% Then $\min \set{f(x) | x \in \R, 0 < x < N} = \frac{a + b + 2\sqrt{ab}}{N}$.
% \end{proposition}

% We notice that when $N$ is large, we can take an integer which is enough close to $\frac{a - \sqrt{ab}}{a - b}N$.

\section{Numerical examples}\label{sec:numerical-examples}
We examine the empirical performances of the all-at-once fitting and the inductive skeleton fitting and verify the asymptotic risks derived in \cref{sec:all-at-once-fitting,sec:inductive-skeleton-fitting} over synthetic instances and multi-objective optimization instances. 
Experiment programs were implemented in Python 3.7.1 and run on a Windows 7 PC with an Intel Core i7-4790CPU (3.60 GHz) and 16 GB RAM.
All experiments are reproducible by the source code and dependent libraries provided in the supplementary materials.

\subsection{Synthetic instances}\label{sec:synthetic-instances}
To verify the asymptotic risks derived in \cref{sec:all-at-once-fitting,sec:inductive-skeleton-fitting}, we consider the fitting problem where the true B\'ezier simplex $\bm b(\bm{t})~(\bm t \in \Delta^{M-1})$ is an $(M-1)$-dimensional unit simplex on $\R^L$, and randomly generate $N$ training points $\set{(\bm t_n, \bm x_n)}_{n = 1}^N$ as $\bm x_n = \bm b(\bm t_n) + \bm{\varepsilon}_n~(\bm \varepsilon_n \sim N(\bm 0, 0.1^2 \bm I))$.
This synthetic instance is parameterized by a tuple $(L, M, N)$. 
%A synthetic instance is generated as follows: given parameters $(\sigma, L, M, N)$, we randomly generated $N$ %parameters $\{\bm t_n\}_{n=1}^N$ on $\Delta^M$ for the all-at-once fitting and the inductive skeleton fitting respectively, and construct 
%training points points , where $\bm x_n$ are defined as .
The detailed data generation processes are shown in the supplementary materials (\cref{apnd:sec:numerical-experiments}).

In this experiment, we estimated the B\'ezier simplex with degree $D = 2$ and 3, and compared the following three fitting methods:
\begin{description}
    \item[all-at-once] the all-at-once fitting (\cref{sec:all-at-once-fitting});
    \item[inductive skeleton (non-optimal)] the inductive skeleton fitting (\cref{sec:inductive-skeleton-fitting}) with $N^{(0)} = \dots = N^{(M - 1)} = N / M$, which does not provide the optimal value of the risk \cref{tab:risk-inductive};
    \item[inductive skeleton (optimal)] the inductive skeleton fitting (\cref{sec:inductive-skeleton-fitting}) where $N^{(0)}, \dots, N^{(M - 1)}$ are determined by minimizing the risk \cref{tab:risk-inductive} under the constraints $\sum_{m = 0}^{M-1} N^{(m)} = N$ and $N^{(m)}\geq 0~(m = 0, \dots, M-1)$. The actual sample size $N^{(m)}$ for each $(D, M)$ are shown in \cref{apnd:sec:numerical-experiments} (\cref{apnd:tab:optimal-subsample-ratio}).
\end{description}
When we calculated an approximation of the expected risk for each method, we randomly chose other 10000 parameters $\set{\bm{\hat t}_n}_{n = 1}^{10000}$ from $U(\Delta^{M - 1})$ as a test set and measured the mean squared error, $\mathrm{MSE} := \frac{1}{10000} \sum_{n = 1}^{10000} \norm{\bm b(\bm{\hat{t}}_n) - \bm{\hat{b}}(\bm{\hat{t}}_n)}^2$, where $\bm{\hat{b}}$ is the estimated B\'ezier simplex.
We ran 20 trials and measured MSEs for each $(L, M, N)$ with $D \in \set{2, 3}$.

Owing to space limitations, we only present typical results here. The remaining results are provided in the supplementary materials (\cref{apnd:sec:numerical-experiments}).
\Cref{fig:MSE-vs-N} shows box plots of MSEs over 20 trials and our theoretical risks \cref{eqn:risk} and \cref{tab:risk-inductive} for each $N \in \set{250, 500, 1000, 2000}$ with $(L, M) = (100, 8)$ and $D \in \set{2, 3}$.
We observe that these figures empirically show that our theoretical risks are correct for both $D = 2$ and 3, and the gap between the actual MSEs and the risks are sufficiently small at $N = 1000$. 
For both $D = 2$ and 3, the inductive skeleton (optimal) always achieved lower MSEs than that of the inductive skeleton (non-optimal). 
This result suggests the efficiency of minimizing the risk (\cref{tab:risk-comparison}) with respect to the sample size of each dimension. 
In addition, the inductive skeleton fitting (optimal) also outperformed the all-at-once fitting in the case of $D = 2$. 
This result also supports the discussion described in \cref{sec:all-vs-inductive}. 
\begin{figure}[h]
 \begin{minipage}{0.49\hsize}
        \centering
    \includegraphics[width=1\textwidth]{neurips2019/fig/D=2_M=8_L=100.pdf}
    \subcaption{D=2}
    \label{fig:MSE-vs-N-D=2}
 \end{minipage}
 \begin{minipage}{0.49\hsize}
        \centering
    \includegraphics[width=1\textwidth]{neurips2019/fig/D=3_M=8_L=100.pdf}
    \subcaption{D=3}
    \label{fig:MSE-vs-N-D=3}
 \end{minipage}
 \caption{Sample size $N$ vs. MSE with $(L, M) = (100, 8)$ (boxplots over 20 trials and theoretical risks).}
 \label{fig:MSE-vs-N}
\end{figure}

\subsection{Multi-objective optimization instances}\label{sec:MOP-instances}
To investigate our results practically, we provide two complementary multi-objective optimization problem instances: a generalized location problem called \texttt{MED} \cite{Harada2006,Hamada2010} and a multi-objective hyper-parameter tuning of the group lasso \cite{Yuan2006} on the \texttt{Birthwt} dataset \cite{Hosmer1989,Venables2002}.
Both of them are strongly convex three-objective optimization problems and we consider fitting their two-dimensional (that is $M = L = 3$) Pareto fronts by a B\'ezier simplex with degree $D = 2$ and 3. % and evaluated the performances.
For the location problem, its Pareto front can be represented by a B\'ezier simplex with degree $D = 2$.
For the group lasso, on the other hand, the Pareto front cannot be represented with degree $D = 2$ but $D = 3$ (see \cref{apnd:sec:Pareto-fronts}).
The detailed description of each problem is shown in \cref{apnd:sec:numerical-experiments}.

As we conducted in the previous experiments, we generated a training set and a test set on a Pareto front randomly then fitted a B\'ezier simplex to the training set and evaluated the MSE between the estimated B\'ezier simplex and the test set. 
We chose the number of training points to $N = 50$ and 100.
With regard to the test set, the number of sample points is 10000 and 1000 for the location problem and the group lasso respectively. 
We repeated experiments 20 times for each $(D, N)$.

For each problem instance and method, the average and the standard deviation of the MSE are shown in \cref{tab:MSE-MOP-instances}. 
In \cref{tab:MSE-MOP-instances}, we highlighted the best score of MSE out of all-at-once fitting and inductive skeleton fitting (optimal) and added the results of one-sided Student's t-test with significance level 0.05.

\begin{table}[h]
\caption{MSE (avg.\ $\pm$ s.d.\ over 20 trials) for the location problem and the group lasso. The winners with significance level $p < 0.05$ are shown in bold.}
\label{tab:MSE-MOP-instances}
\begin{minipage}[t]{.5\textwidth}
    \centering
    \subcaption{Location problem}
    \scriptsize
    {\tabcolsep = 1.5mm
    \begin{tabular}{ccll}
    \toprule
$D$&$N$ & \multicolumn{1}{c}{All-at-once} & \multicolumn{1}{c}{Inductive-skeleton (optimal)}\\ \midrule
%2&25	&3.582e-04 $\pm$	6.703e-05	&\textbf{3.126e-04 $\pm$	4.375e-05}$^{**}$\\
2&50	&2.855e-04 $\pm$	2.114e-05	&\textbf{2.691e-04 $\pm$	8.541e-06}\\
&100 &2.660e-04 $\pm$	1.227e-05	&\textbf{2.608e-04 $\pm$	5.946e-06}\\ \midrule
%3&25	&7.502e-04 $\pm$	5.352e-04	&\textbf{4.626e-04 $\pm$	1.786e-04}$^{**}$\\
3&50	&3.596e-04 $\pm$	7.935e-05	&3.269e-04 $\pm$	3.969e-05\\ 
&100	&2.810e-04 $\pm$	1.569e-05	&2.796e-04 $\pm$	1.478e-05\\
\bottomrule
    \end{tabular}
    }
    \label{tab:location-problem}
\end{minipage}
\begin{minipage}[t]{.5\textwidth}
    \centering
    \subcaption{Group lasso}
    \scriptsize
    {\tabcolsep = 1.5mm
    \begin{tabular}{ccll}
    \toprule
$D$&$N$ & \multicolumn{1}{c}{All-at-once} & \multicolumn{1}{c}{Inductive-skeleton (optimal)}\\ \midrule
%2&25	&\textbf{1.256e-04 $\pm$	 4.023e-05}$^{**}$	&5.123e-04 $\pm$	3.546e-05\\
2&50	&\textbf{1.041e-04 $\pm$	 1.614e-05}	&4.966e-04 $\pm$	1.848e-05\\
& 100 &\textbf{8.949e-05 $\pm$	6.083e-06}	&5.020e-04 $\pm$	1.276e-05\\ \midrule
%3&25	&\textbf{1.066e-04 $\pm$	1.182e-04}	&1.813e-04 $\pm$ 	2.132e-04\\
3&50	&\textbf{4.354e-05 $\pm$	 1.526e-05}	&1.206e-04 $\pm$ 	9.440e-06\\
& 100 &\textbf{3.231e-05 $\pm$	8.058e-06}	&1.141e-04 $\pm$	8.200e-06\\
 \bottomrule
    \end{tabular}
    }
    \label{tab:group-lasso}
\end{minipage}
\end{table}

Since the Pareto front of the location problem can be represented by a B\'ezier simplex of $D = 2$ and 3, we expected that the experimental results agree with our analysis discussed in \cref{sec:all-vs-inductive}.
In fact, \cref{tab:location-problem} shows that the inductive skeleton (optimal) outperformed for $D = 2$, which is consistent with our analysis.
For $D = 3$, the difference of MSEs is not significant.
\Cref{tab:risk-comparison} suggests that the difference of the risks between the two methods is very small for $(D, M) = (3, 3)$, and thus we did not observe significant differences of MSEs for $N = 50$ and 100.

%In case of MED, the inductive skeleton (optimal) always achieved better MSEs than that of the all-at-once and the difference between the two methods is significant for $D=2$.
%For $D=3$ however, there is no significant difference.

In case of the group lasso, on the other hand, \cref{tab:group-lasso} shows that the all-at-once was better for both $D = 2$ and 3, and the differences are all significant.
While our analysis assumes that the target hypersurface to be fitted can be represented by a B\'ezier simplex, the Pareto front of the group lasso cannot for $D = 2$ but for $D = 3$.
Therefore, the results for $D = 2$ does not contradict to our analysis.
Moreover, the results for $D = 3$ that the all-at-once achieved better MSEs accords with our analysis.

%The reason why the inductive skeleton (optimal) did not outperform for $D=2$ is that the Pareto front of Birthwt originally cannot be represented by a B\'ezier simplex of degree two.

From the above results, the validity of the analytic results is confirmed in practical situations.

\section{Conclusion}\label{sec:conclusion}
In this paper, we have shown that the asymptotic $\ell_2$-risk of the two B\'ezier simplex fitting methods developed previously: the all-at-once fitting and the inductive skeleton fitting.
From our risk analysis, the optimal ratio of subsamples for the inductive skeleton fitting has been derived, which is useful for design of experiments to maximize the goodness of fit.
We have discussed that superiority between the two fitting methods depends on the degree of a B\'ezier simplex to be fit: the inductive skeleton fitting with optimally-decoupled subsamples outperforms for degree two whereas the all-at-once fitting becomes the better for degree three, independent of the dimensionality of the B\'ezier simplex and its ambient space.
The above theoretical results have been confirmed via numerical experiments under small to moderate sample sizes.
We have demonstrated two applications of the analytic results in multi-objective optimization: a generalized location problem and a hyper-parameter tuning of the group lasso.

As a remark for future work, we point out two important cases which the current theory does not cover.
The first one is the case discussed in \cref{sec:MOP-instances} that the true surface is not representable by a model.
The second one is presented in the literature \cite{Kobayashi2019}.
When the parameters of a B\'ezier simplex are not given in a sample and to be estimated as well as the control points, the inductive skeleton fitting outperforms the all-at-once fitting even if the B\'ezier simplex is of degree \emph{three}.
We believe that those cases would offer insightful examples to extend the scope of the theory.

\bibliographystyle{plain}
\bibliography{reference}
\end{document}
